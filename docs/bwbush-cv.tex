\documentclass[]{article}
\usepackage{lmodern}
\usepackage{amssymb,amsmath}
\usepackage{ifxetex,ifluatex}
\usepackage{fixltx2e} % provides \textsubscript
\ifnum 0\ifxetex 1\fi\ifluatex 1\fi=0 % if pdftex
  \usepackage[T1]{fontenc}
  \usepackage[utf8]{inputenc}
\else % if luatex or xelatex
  \ifxetex
    \usepackage{mathspec}
    \usepackage{xltxtra,xunicode}
  \else
    \usepackage{fontspec}
  \fi
  \defaultfontfeatures{Mapping=tex-text,Scale=MatchLowercase}
  \newcommand{\euro}{€}
\fi
% use upquote if available, for straight quotes in verbatim environments
\IfFileExists{upquote.sty}{\usepackage{upquote}}{}
% use microtype if available
\IfFileExists{microtype.sty}{\usepackage{microtype}}{}
\ifxetex
  \usepackage[setpagesize=false, % page size defined by xetex
              unicode=false, % unicode breaks when used with xetex
              xetex]{hyperref}
\else
  \usepackage[unicode=true]{hyperref}
\fi
\hypersetup{breaklinks=true,
            bookmarks=true,
            pdfauthor={},
            pdftitle={},
            colorlinks=true,
            citecolor=blue,
            urlcolor=blue,
            linkcolor=magenta,
            pdfborder={0 0 0}}
\urlstyle{same}  % don't use monospace font for urls
\setlength{\parindent}{0pt}
\setlength{\parskip}{6pt plus 2pt minus 1pt}
\setlength{\emergencystretch}{3em}  % prevent overfull lines
\setcounter{secnumdepth}{0}


\begin{document}

\section{Brian W Bush -- Curriculum
Vitae}\label{brian-w-bush-curriculum-vitae}

650 Hartford Dr\\Boulder, CO 80305 USA

720--258--6728
(voice)\\\href{mailto:b.w.bush@acm.org}{b.w.bush@acm.org}\\\url{http://www.bwbush.io/}

\begin{center}\rule{3in}{0.4pt}\end{center}

\subsubsection{Mission}\label{mission}

To research and develop high-quality, integrated analyses and analysis
techniques for problems important to society through scientific
computing, computational physics, data analysis/mining, and the modeling
\& simulation of complex systems.

\subsubsection{Vision}\label{vision}

To be a member of a quality- and productivity-oriented team performing
innovative, ground-breaking research and development in scientific
computation for complex systems.

\subsubsection{Objective}\label{objective}

A research \& development or project leadership position in computing
(especially scientific computing, the modeling and simulation of complex
systems, data analysis/mining, or computational physics), as a member of
a quality- and productivity-oriented team.

\subsubsection{Biographical Sketch}\label{biographical-sketch}

Brian W. Bush is a Principal Engineer in the
\href{http://www.nrel.gov/analysis/staff_efm.html}{Energy Forecasting \&
Modeling Group} of the
\href{http://www.nrel.gov/analysis/about_office.html}{Strategic Energy
Analysis Center} at the \href{http://www.nrel.gov/}{National Renewable
Energy Laboratory {[}NREL{]}}. He currently leads NREL's
\href{projects/bsm.html}{Biomass Scenario Model {[}BSM{]} project}, a
system-dynamics simulation of the cellulosic biomass-to-biofuels supply
chain, and its \href{projects/sera.html}{Scenario Evaluation and
Regionalization Analysis {[}SERA{]} project}, an optimization tool for
regional hydrogen infrastructure. He has also led and collaborated on
projects focused on numerous data-science and data-engineering projects
such as the geostatistical modeling and analysis of the spatial and
temporal correlations in renewable energy resources, automated anomaly
detection and correction in energy datasets, and logical reasoning on
qualitative semi-structured databases. He is a member of NREL's Research
Council.

For more than seventeen years prior to his arrival at NREL, he was a
technical staff member in the Energy and Infrastructure Analysis Group
at Los Alamos National Laboratory {[}LANL{]}. His work on
\href{projects/transims.html}{TRANSIMS {[}the Transportation Analysis
Simulation System{]}} from 1994 to 2000 there focused on leading
research on software architecture, the representation of road networks,
microsimulation output collection, and data compression. More recently,
he developed computer simulations for complex phenomena such as
\href{projects/infrastructure.html}{interacting critical
infrastructures} and \href{projects/a-la-carte.html}{supercomputer
hardware architectures}. He formerly directed the
\href{projects/ieiss.html}{Interdependent Energy Infrastructure
Simulation System {[}IEISS{]}} and \href{projects/cipdss.html}{Critical
Infrastructure Protection Decision Support System {[}CIPDSS{]}}
projects, leading teams of approximately one dozen people, and held the
position of Thrust Area Leader for the U.S. Dept. of Homeland Security's
Critical Infrastructure Protection Portfolio in its Science \&
Technology Directorate. At LANL he was a member of its Patent Committee
and its Institutional Computing Technical Committee. As a visiting
scientist at the National Center for Atmospheric Research (NCAR), he
initiated efforts to \href{projects/weather-impacts.html}{connect
simulations of weather and climate to impact models for energy and
infrastructure networks}.

Brian received a B.S. in physics at the California Institute of
Technology in 1985 and a Ph.D.~in theoretical physics at Yale University
in 1990, where he was a National Science Foundation graduate fellow. He
extensively performed and published \href{projects/physics.html}{physics
research on ultrarelativistic heavy-ion collisions and nuclear shape
fluctuations}. His research interests include statistical physics,
symbolic dynamics, information theory, data mining, approximate
reasoning, theory of simulation, and visualization.

\begin{center}\rule{3in}{0.4pt}\end{center}

\subsection{Skills}\label{skills}

\subsubsection{Project Leadership}\label{project-leadership}

Successfully led complex software- and data-intensive research \&
development teams of a dozen people.

Experience writing proposals for the NSF, DOE, DOD, and DHS.

\subsubsection{Research}\label{research}

Broad and innovative research, analysis, modeling, and simulation
experience with complex systems and in disciplines ranging from
transportation research to graph theory.

Data mining, statistical learning, approximate reasoning, and
visualization expertise with massive data.

Strong theoretical physics basic research background, involving both
analytic and numerical work.

\subsubsection{Computing}\label{computing}

Extensive scientific computing abilities (particularly simulation
architecture, design, and implementation) and numerical
methods/algorithms knowledge.

Comprehensive experience with object-oriented software engineering:
analysis, architecture, design, coding, testing; Booch methodology and
UML.

Design and development expertise with system architectures,
frameworks/patterns/libraries/wrappers, data structures, and algorithms.

\emph{Languages:} active use of C++ (1991--present), Haskell
(2002--present), Java (1996--present), SQL (1992--present), XML/XSL
(1998--present); familiarity with Prolog (1993--present); past use of
Eiffel (1994--1997), FORTRAN (1979--1991), Pascal (1980--1985), PL/I
(1982), POSTSCRIPT (1990--1992), SGML/DSSSL (1996--2001), Smalltalk
(1993--1996), VRML (1996--2001).

\emph{Operating Systems:} UNIX (1990--present), Linux clusters
(1998--present), Windows (1992--present).

\emph{Software:} databases---MySQL, Oracle, PostgreSQL, Derby;
services---Globus, WSDL, WS-*, SOAP; geographic information
systems---ArcGIS, GeoTools, WFS/WMS/GML; class libraries---STL, MPI,
OpenMap, Boost, Eclipse RCP/EMF/Ecore;
mathematics/statistics/visualization---Mathematica, R, GGobi; software
engineering---SourceForge, Rational Rose, Together, Eclipse;
configuration management---ClearCase, CVS, TeamWare, Subversion, GIT,
Mercurial; simulation---DaSSF, Vensim, STELLA; ontology---OWL, RDF,
Protégé, TopBraid Composer.

\emph{Application Areas:} discrete-event and continuous simulation;
graph and network analysis; mathematical and statistical modeling; data
analysis, mining, and visualization; productivity tools.

\subsubsection{Languages}\label{languages}

Ability to read modern Italian, classical Latin, classical Greek, and
Pāli.

\begin{center}\rule{3in}{0.4pt}\end{center}

\subsection{Education}\label{education}

\subsubsection{Ph.D.~in Theoretical Physics, December 1990 {[}GPA
3.8{]}}\label{ph.d.in-theoretical-physics-december-1990-gpa-3.8}

Yale University, New Haven, CT

\emph{Curriculum:} atomic, nuclear, and high energy physics; field
theory, including the standard model; scattering theory; group theory
and mathematical methods.

\emph{Dissertation Title:} Shape Fluctuations in Hot Rotating Nuclei.

\subsubsection{B.S. in Physics (with honor), June 1985 {[}GPA
4.0{]}}\label{b.s.-in-physics-with-honor-june-1985-gpa-4.0}

California Institute of Technology, Pasadena, CA

\emph{Curriculum:} classical, quantum, and statistical physics;
mathematical methods, including abstract algebra, complex analysis,
asymptotics, and spectral theory; probability and statistics.

\emph{Senior Thesis:} set upper limits on proton lifetime in several
decay modes using IMB collaboration experimental data.

\subsubsection{Continuing Education}\label{continuing-education}

\emph{Subject Areas:} statistics, graph theory, fuzzy logic, information
theory, genetic algorithms, neural networks, data mining, approximate
reasoning, simulation; visualization; object-oriented analysis and
design, data structures, algorithms, patterns, ontologies, effective
user interface design, software testing and quality assurance, software
engineering process; project leadership \& facilitation.

``Learning from Data,'' California Institute of Technology, Spring 2012.

``Proposing to Win,'' Korn/Ferry International, 10--11 January 2012.

``Intermediate Dynamic Modeling with STELLA and iThink,'' ISEE Systems,
10--12 December 2008.

``Software Inspections and Peer Reviews,'' Los Alamos National
Laboratory, 28 February 2006.

``Presenting Data and Information,'' Yale University, 28 July 2003.

``Statistical Learning and Data Mining: from Supervised to Unsupervised
Learning,'' Stanford University, 6--7 September 2001.

``Modern Regression and Classification,'' Stanford University, 31
January--1 February 2000.

``Mastering Projects,'' True North pgs, 8--10 June 1999.

``Data Mining: Principles and Practice,'' Gordian Institute, 4--6
November 1998.

``Substation Communications and Automation,'' University of Wisconsin --
Madison, College of Engineering, 26--28 June 1996.

``Engineering Simulation using SimSmart,'' Advance High Technology
Corporation, 11--15 December 1995.

``Software Testing and Quality Assurance,'' University of California,
Berkeley, University Extension, 18--20 October 1995.

``Effective GUI Design,'' Corporate Computing International, 23--24 May
1995.

``Object-Oriented Analysis and Design with C++,'' Catalyst Solutions, 30
August--2 September 1994.

\subsubsection{Standardized Tests}\label{standardized-tests}

\emph{SAT (1981):} combined -- 1470 (99th percentile); math -- 750 (99th
percentile); verbal -- 720 (99th percentile).

\emph{GRE (1984):} quantitative -- 800 (99th percentile); analytical --
660; verbal -- 720 (96th percentile); physics -- 950 (97th percentile).

\begin{center}\rule{3in}{0.4pt}\end{center}

\subsection{Work Experience}\label{work-experience}

\subsubsection{Principal Engineer,
2008--present}\label{principal-engineer-2008present}

National Renewable Energy Laboratory, Golden, CO

Led the Biomass Scenario Model (BSM) project, a system-dynamics
simulation of the cellulosic biomass-to-biofuels supply chain. Developed
an enhanced version of the Scenario Evaluation and Regionalization
Analysis (SERA), an optimization tool for regional hydrogen
infrastructure. Also collaborated on statistical and geospatial analyses
of renewable energy systems (photovoltaics and wind-turbine farms) from
technical, financial, and economic perspectives. Researched issues
around supporting community decisions in their transition to more
efficient and renewable energy use. Led a data-mining and
analysis-automation project that is developing new statistical and
machine learning methods for application to renewable energy
information.

\subsubsection{Technical Staff Member,
1993--2008}\label{technical-staff-member-19932008}

Los Alamos National Laboratory, Los Alamos, NM

Led numerous innovative research-oriented simulation projects (often
several simultaneously) with teams of approximately one dozen people and
with an emphasis on robust architecture and software quality.
Collaborated as a software architect, designer, and programmer on
TRANSIMS, a multiyear, multimillion-dollar project to develop a
transportation simulation system; led team for information and data
handling research and development. Performed ground-breaking research on
information theory, graph theory, and infrastructure (electric power,
control communications, interdependence) assurance; architected,
designed, and developed simulation software, analysis software
applications, geographic information systems, and relational databases.
As a Visiting Scientist at the National Center for Atmospheric Research
(NCAR), recently began efforts to connect simulations of weather and
climate to impact models for energy and infrastructure networks.

\subsubsection{Director-Funded Postdoctoral Fellow,
1990--1992}\label{director-funded-postdoctoral-fellow-19901992}

Los Alamos National Laboratory, Los Alamos, NM

Developed theory and computer simulations of ultrarelativistic heavy-ion
collisions, nuclear dissipation, shape diffusion, and level densities,
and solvable nuclear models.

\subsubsection{Research Associate, 1987--1992 (consultant), and Research
Assistant,
1984--1987}\label{research-associate-19871992-consultant-and-research-assistant-19841987}

Pacific-Sierra Research Corporation, W. Los Angeles, CA

Analyzed and developed computer simulations of large area urban fires,
wildland fires, ignition phenomena; studied atmospheric and
environmental impacts of nuclear weapons.

\subsubsection{Research Assistant, 1988--1990, and National Science
Foundation Graduate Fellow,
1985--1988}\label{research-assistant-19881990-and-national-science-foundation-graduate-fellow-19851988}

Yale University, New Haven, CT

Researched nuclear shape fluctuations, phase transitions, giant
resonances, and Landau theory in hot rotating nuclei, comparing
numerically computed predictions to available experimental data;
investigated the shell model, level densities, nuclear damping, and
pion-proton scattering.

\subsubsection{Teaching Assistant,
1985--1987}\label{teaching-assistant-19851987}

Yale University, New Haven, CT

Assisted teaching laboratory courses for science majors, physics majors,
and physics graduate students.

\subsubsection{Research Assistant, 1984}\label{research-assistant-1984}

California Institute of Technology, Pasadena, CA

Assisted development of Monte-Carlo computer simulations for muon-decay
parity-violation experiment.

\subsubsection{Materiel Assistant, 1982}\label{materiel-assistant-1982}

Hughes Aircraft Company, Electro-Optical and Data Systems Group, El
Segundo, CA

Developed computer programs for business applications related to
procurement.

\subsubsection{Assistant to Accounts Receivable,
1981}\label{assistant-to-accounts-receivable-1981}

Bel-Air Bay Club, Pacific Palisades, CA

Performed various clerical tasks for accounts receivable and payroll.

\begin{center}\rule{3in}{0.4pt}\end{center}

\subsection{Other Experience}\label{other-experience}

\subsubsection{Professional Society
Membership}\label{professional-society-membership}

American Physical Society, 1982--1985, 1995--present.

Association for Computing Machinery, 1996--present.

Institute of Electrical and Electronics Engineers (including Computer
Society), 1995--present.

Society for Industrial and Applied Mathematics, 1995--2008.

Society for Risk Analysis, 2004--2005.

\subsubsection{Honors}\label{honors}

National Renewable Energy Laboratory President's Award, 2013.

National Renewable Energy Laboratory FY2010 Outstanding Achievement
Award, 2011.

Los Alamos Achievement Award, 2004.

Los Alamos Achievement Award, 2001.

University of California and U.S. Department of Energy Distinguished
Copyright Award, 2000.

Los Alamos Achievement Award, 1996 (twice).

Sterling Prize Fellowship, 1988.

National Science Foundation Graduate Fellow, 1985--88.

J. W. Gibbs Fellowship, 1985.

National Merit Scholar finalist, 1981.

Bank of America Achievement Award, 1981.

California Junior Classical League Competition, first place in 3rd year
Latin Grammar, 1981.

Bay Area Math League Competition, fourth place, 1981.

California Scholastic Federation member, 1979--81.

Rensselaer Polytechnic Institute medallist, 1980.

California Junior Classical League Competition, first place in 2nd year
Latin Grammar, 1980.

Occidental College Math Field Day, finalist, 1980.

California High School Cooperative Chemistry Exam, first place, 1980.

California Junior Classical League Competition, first place in 1st year
Latin Grammar, 1979.

\subsubsection{Collaboration}\label{collaboration}

Collaborator with scientists at Argonne National Laboratory, Brookhaven
National Laboratory, Idaho National Engineering Laboratory, Los Alamos
National Laboratory, Michigan State University, National Center for
Atmospheric Research, Oak Ridge National Laboratory, Pacific Northwest
National Laboratory, Sandia National Laboratories, University of
Colorado, University of Denver, University of Washington, Washington
State University, Weizmann Institute, and Yale University.

\subsubsection{Teaching}\label{teaching}

Frequently mentored graduate students and post-docs.

Supervised graduate research assistants.

Tutored math and writing in primary and secondary schools.

Occasionally taught or assisted teaching university courses up to
graduate level.

\subsubsection{Software Authorship
Metrics}\label{software-authorship-metrics}

C++: 173,127 source lines of code.

FORTRAN: 17,868 source lines of code.

Haskell: 20,215 source lines of code.

Java: 186,400 source lines of code.

Smalltalk: 30,664 source lines of code.

\subsubsection{Beta Testing}\label{beta-testing}

Tested pre-release versions of software for a major commercial database
and programming language product firm, 1994--1996.

\subsubsection{Committee Membership}\label{committee-membership}

Research Council, National Renewable Energy Laboratory, 2014--present.

Patent Committee at Los Alamos National Laboratory, 2002--2005.

Institutional Computing Technical Committee at Los Alamos National
Laboratory, 2002--2003.

\end{document}