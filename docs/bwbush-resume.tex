\documentclass[]{article}
\usepackage{lmodern}
\usepackage{amssymb,amsmath}
\usepackage{ifxetex,ifluatex}
\usepackage{fixltx2e} % provides \textsubscript
\ifnum 0\ifxetex 1\fi\ifluatex 1\fi=0 % if pdftex
  \usepackage[T1]{fontenc}
  \usepackage[utf8]{inputenc}
\else % if luatex or xelatex
  \ifxetex
    \usepackage{mathspec}
    \usepackage{xltxtra,xunicode}
  \else
    \usepackage{fontspec}
  \fi
  \defaultfontfeatures{Mapping=tex-text,Scale=MatchLowercase}
  \newcommand{\euro}{€}
\fi
% use upquote if available, for straight quotes in verbatim environments
\IfFileExists{upquote.sty}{\usepackage{upquote}}{}
% use microtype if available
\IfFileExists{microtype.sty}{\usepackage{microtype}}{}
\usepackage{geometry}
\geometry{letterpaper, portrait, margin=0.75in}
\ifxetex
  \usepackage[setpagesize=false, % page size defined by xetex
              unicode=false, % unicode breaks when used with xetex
              xetex]{hyperref}
\else
  \usepackage[unicode=true]{hyperref}
\fi
\hypersetup{breaklinks=true,
            bookmarks=true,
            pdfauthor={},
            pdftitle={},
            colorlinks=true,
            citecolor=blue,
            urlcolor=blue,
            linkcolor=magenta,
            pdfborder={0 0 0}}
\urlstyle{same}  % don't use monospace font for urls
\setlength{\parindent}{0pt}
\setlength{\parskip}{6pt plus 2pt minus 1pt}
\setlength{\emergencystretch}{3em}  % prevent overfull lines
\setcounter{secnumdepth}{0}


\begin{document}

\section{Brian W Bush}\label{brian-w-bush}

650 Hartford Dr\\Boulder, CO 80305 USA

720--258--6728
(voice)\\\href{mailto:b.w.bush@acm.org}{b.w.bush@acm.org}\\\url{http://www.bwbush.io/}

\begin{center}\rule{3in}{0.4pt}\end{center}

\subsubsection{Objective}\label{objective}

A research \& development or project leadership position in computing
(especially scientific computing, the modeling and simulation of complex
systems, data analysis/mining, or computational physics), as a member of
a quality- and productivity-oriented team.

\begin{center}\rule{3in}{0.4pt}\end{center}

\subsection{Skills}\label{skills}

\subsubsection{Project Leadership}\label{project-leadership}

Successfully led complex software- and data-intensive research \&
development teams of a dozen people.

Experience writing proposals for the NSF, DOE, DOD, and DHS.

\subsubsection{Research}\label{research}

Broad and innovative research, analysis, modeling, and simulation
experience with complex systems and in disciplines ranging from
transportation research to graph theory.

Data mining, statistical learning, approximate reasoning, and
visualization expertise with massive data.

Strong theoretical physics basic research background, involving both
analytic and numerical work.

\subsubsection{Computing}\label{computing}

Extensive scientific computing abilities (particularly simulation
architecture, design, and implementation) and numerical
methods/algorithms knowledge.

Comprehensive experience with object-oriented software engineering:
analysis, architecture, design, coding, testing; Booch methodology and
UML.

Design and development expertise with system architectures,
frameworks/patterns/libraries/wrappers, data structures, and algorithms.

\emph{Languages:} active use of C++ (1991--present), Haskell
(2002--present), Java (1996--present), SQL (1992--present), XML/XSL
(1998--present); familiarity with Prolog (1993--present); past use of
Eiffel (1994--1997), FORTRAN (1979--1991), Pascal (1980--1985), PL/I
(1982), POSTSCRIPT (1990--1992), SGML/DSSSL (1996--2001), Smalltalk
(1993--1996), VRML (1996--2001).

\emph{Operating Systems:} UNIX (1990--present), Linux clusters
(1998--present), Windows (1992--present).

\emph{Software:} databases---MySQL, Oracle, PostgreSQL, Derby;
services---Globus, WSDL, WS-*, SOAP; geographic information
systems---ArcGIS, GeoTools, WFS/WMS/GML; class libraries---STL, MPI,
OpenMap, Boost, Eclipse RCP/EMF/Ecore;
mathematics/statistics/visualization---Mathematica, R, GGobi; software
engineering---SourceForge, Rational Rose, Together, Eclipse;
configuration management---ClearCase, CVS, TeamWare, Subversion, GIT,
Mercurial; simulation---DaSSF, Vensim, STELLA; ontology---OWL, RDF,
Protégé, TopBraid Composer.

\emph{Application Areas:} discrete-event and continuous simulation;
graph and network analysis; mathematical and statistical modeling; data
analysis, mining, and visualization; productivity tools.

\begin{center}\rule{3in}{0.4pt}\end{center}

\subsection{Education}\label{education}

\subsubsection{Ph.D.~in Theoretical Physics, December 1990 {[}GPA
3.8{]}}\label{ph.d.in-theoretical-physics-december-1990-gpa-3.8}

Yale University, New Haven, CT

\emph{Curriculum:} atomic, nuclear, and high energy physics; field
theory, including the standard model; scattering theory; group theory
and mathematical methods.

\emph{Dissertation Title:} Shape Fluctuations in Hot Rotating Nuclei.

\subsubsection{B.S. in Physics (with honor), June 1985 {[}GPA
4.0{]}}\label{b.s.-in-physics-with-honor-june-1985-gpa-4.0}

California Institute of Technology, Pasadena, CA

\emph{Curriculum:} classical, quantum, and statistical physics;
mathematical methods, including abstract algebra, complex analysis,
asymptotics, and spectral theory; probability and statistics.

\emph{Senior Thesis:} set upper limits on proton lifetime in several
decay modes using IMB collaboration experimental data.

\subsubsection{Continuing Education}\label{continuing-education}

\emph{Subject Areas:} statistics, graph theory, fuzzy logic, information
theory, genetic algorithms, neural networks, data mining, approximate
reasoning, simulation; visualization; object-oriented analysis and
design, data structures, algorithms, patterns, ontologies, effective
user interface design, software testing and quality assurance, software
engineering process; project leadership \& facilitation.

\begin{center}\rule{3in}{0.4pt}\end{center}

\subsection{Work Experience}\label{work-experience}

\subsubsection{Principal Engineer,
2008--present}\label{principal-engineer-2008present}

National Renewable Energy Laboratory, Golden, CO

Led the Biomass Scenario Model (BSM) project, a system-dynamics
simulation of the cellulosic biomass-to-biofuels supply chain. Developed
an enhanced version of the Scenario Evaluation and Regionalization
Analysis (SERA), an optimization tool for regional hydrogen
infrastructure. Also collaborated on statistical and geospatial analyses
of renewable energy systems (photovoltaics and wind-turbine farms) from
technical, financial, and economic perspectives. Researched issues
around supporting community decisions in their transition to more
efficient and renewable energy use. Led a data-mining and
analysis-automation project that is developing new statistical and
machine learning methods for application to renewable energy
information.

\subsubsection{Technical Staff Member,
1993--2008}\label{technical-staff-member-19932008}

Los Alamos National Laboratory, Los Alamos, NM

Led numerous innovative research-oriented simulation projects (often
several simultaneously) with teams of approximately one dozen people and
with an emphasis on robust architecture and software quality.
Collaborated as a software architect, designer, and programmer on
TRANSIMS, a multiyear, multimillion-dollar project to develop a
transportation simulation system; led team for information and data
handling research and development. Performed ground-breaking research on
information theory, graph theory, and infrastructure (electric power,
control communications, interdependence) assurance; architected,
designed, and developed simulation software, analysis software
applications, geographic information systems, and relational databases.
As a Visiting Scientist at the National Center for Atmospheric Research
(NCAR), recently began efforts to connect simulations of weather and
climate to impact models for energy and infrastructure networks.

\subsubsection{Director-Funded Postdoctoral Fellow,
1990--1992}\label{director-funded-postdoctoral-fellow-19901992}

Los Alamos National Laboratory, Los Alamos, NM

Developed theory and computer simulations of ultrarelativistic heavy-ion
collisions, nuclear dissipation, shape diffusion, and level densities,
and solvable nuclear models.

\subsubsection{Research Associate, 1987--1992 (consultant), and Research
Assistant,
1984--1987}\label{research-associate-19871992-consultant-and-research-assistant-19841987}

Pacific-Sierra Research Corporation, W. Los Angeles, CA

Analyzed and developed computer simulations of large area urban fires,
wildland fires, ignition phenomena; studied atmospheric and
environmental impacts of nuclear weapons.

\subsubsection{Research Assistant, 1988--1990, and National Science
Foundation Graduate Fellow,
1985--1988}\label{research-assistant-19881990-and-national-science-foundation-graduate-fellow-19851988}

Yale University, New Haven, CT

Researched nuclear shape fluctuations, phase transitions, giant
resonances, and Landau theory in hot rotating nuclei, comparing
numerically computed predictions to available experimental data;
investigated the shell model, level densities, nuclear damping, and
pion-proton scattering.

\begin{center}\rule{3in}{0.4pt}\end{center}

\subsection{Other Experience}\label{other-experience}

\subsubsection{Publication}\label{publication}

Total of 185 publications (cited over 1000 times), including 30 in
refereed journals such as \emph{Science}, \emph{PLOS ONE},
\emph{Physical Review Letters}, and \emph{Fuzzy Sets \& Systems}, 28 in
conference proceedings, 10 conference posters, three in submission
process, one patent disclosure, and 101 technical reports. Software
packages written in C++ (173k+ SLOC), FORTRAN (15k+ SLOC), Haskell (20k+
SLOC), Java (186k+ SLOC), and Smalltalk (30k+ SLOC).

\subsubsection{Public Speaking}\label{public-speaking}

Speaker 16 times at scientific conferences/workshops and 73 times in
seminar series; participant at seven other scientific and engineering
conferences.

\end{document}
