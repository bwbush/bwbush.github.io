\documentclass[]{article}
\usepackage{lmodern}
\usepackage{amssymb,amsmath}
\usepackage{ifxetex,ifluatex}
\usepackage{fixltx2e} % provides \textsubscript
\ifnum 0\ifxetex 1\fi\ifluatex 1\fi=0 % if pdftex
  \usepackage[T1]{fontenc}
  \usepackage[utf8]{inputenc}
\else % if luatex or xelatex
  \ifxetex
    \usepackage{mathspec}
    \usepackage{xltxtra,xunicode}
  \else
    \usepackage{fontspec}
  \fi
  \defaultfontfeatures{Mapping=tex-text,Scale=MatchLowercase}
  \newcommand{\euro}{€}
\fi
% use upquote if available, for straight quotes in verbatim environments
\IfFileExists{upquote.sty}{\usepackage{upquote}}{}
% use microtype if available
\IfFileExists{microtype.sty}{\usepackage{microtype}}{}
\ifxetex
  \usepackage[setpagesize=false, % page size defined by xetex
              unicode=false, % unicode breaks when used with xetex
              xetex]{hyperref}
\else
  \usepackage[unicode=true]{hyperref}
\fi
\hypersetup{breaklinks=true,
            bookmarks=true,
            pdfauthor={},
            pdftitle={},
            colorlinks=true,
            citecolor=blue,
            urlcolor=blue,
            linkcolor=magenta,
            pdfborder={0 0 0}}
\urlstyle{same}  % don't use monospace font for urls
\setlength{\parindent}{0pt}
\setlength{\parskip}{6pt plus 2pt minus 1pt}
\setlength{\emergencystretch}{3em}  % prevent overfull lines
\setcounter{secnumdepth}{0}


\begin{document}

\section{Brian W Bush -- Publications}\label{brian-w-bush-publications}

\subsection{Preprints}\label{preprints}

B. W. Bush, E. Doris, and D. Getman, ``Understanding the Complexities of
Subnational Incentives in Supporting a National Market for Distributed
PV.'' The purpose of this work is to contribute to the discussion of the
impact subnational (State, local, and utility) policies, as a group,
have on the national market for solar PV technologies. Subnational
policies have increased in volume in recent years and federal incentives
are set to be phased out over the next few. Understanding how
subnational policies intertwine within and across jurisdictions to
impact market development provides information to both federal program
administrators and subnational policymakers to support informed policy
decision making. In order to do add value to the current discussion on
subnational policies, a logic engine containing incentives for PV
development was developed. The incentives used are from the US
Department of Energy's (DOE) Database of Incentives for Energy
Efficiency and Renewable Energy (DSIRE), widely considered to be the
most comprehensive listing of US policies and incentives. The logic
engine adds value to the DSIRE database by allowing users to see the
interactions of incentives along with exogenous factors such as
geographic location, technology characteristics, and financial
parameters. Depending on how it is queried, the database can yield
insights into which combinations of incentives are available and most
advantageous under particular circumstances as requested by the user. To
illustrate the capabilities of the logic engine, the database was
queried to identify the relative complexities of incentives at the
subnational level. The goal was to identify how many incentives they
qualify for, and the subset of those that result in the largest monetary
benefit (the ``best'' combination of incentives). The outcomes inform
federal program designers as to the complexity of the incentive
``market'' nationwide, and subnational policymakers to the complexities
of incentives within their jurisdiction, as well as allow for a
comparison of complexities between jurisdictions.

M. S. Melaina, Y. Sun, and B. Bush, ``Retail Infrastructure Cost
Comparison for Hydrogen and Electricity Light-Duty Vehicles,'' presented
at the SAE 2014 World Congress and Exhibition, Detroit, Michigan. Both
hydrogen and plug-in electric vehicles offer significant social benefits
to enhance energy security and reduce criteria and greenhouse gas
emissions from the transportation sector. However, the rollout of
electric vehicle supply equipment (EVSE) and hydrogen retail stations
(HRS) requires substantial investments with high risks due to many
uncertainties. We compare retail infrastructure costs on a common basis
-- cost per mile, assuming fueling service to 10\% of all light-duty
vehicles in a typical 1.5 million person city in 2025. Our analysis
considers three HRS sizes, four distinct types of EVSE and two distinct
EVSE scenarios. EVSE station costs, including equipment and
installation, are assumed to be 15\% less than today's costs. We find
that levelized retail capital costs per mile are essentially
indistinguishable given the uncertainty and variability around input
assumptions. Total fuel costs per mile for battery electric vehicle
(BEV) and plug-in hybrid vehicle (PHEV) are, respectively, 21\% lower
and 13\% lower than that for hydrogen fuel cell electric vehicle (FCEV)
under the home-dominant scenario. Including fuel economies and vehicle
costs makes FCEVs and BEVs comparable in terms of costs per mile, and
PHEVs are about 10\% less than FCEVs and BEVs. To account for geographic
variability in energy prices and hydrogen delivery costs, we use the
Scenario Evaluation, Regionalization and Analysis (SERA) model and
confirm the aforementioned estimate of cost per mile, nationally
averaged, but see a 15\% variability in regional costs of FCEVs and a
5\% variability in regional costs for BEVs.

L. J. Vimmerstedt, B. W. Bush, D. D. Hsu, D. Inman, and S. Peterson,
``Maturation of Biomass-to-Biofuels Conversion Technology Pathways for
Rapid Expansion of Biofuels Production: A System Dynamics Perspective,''
Biofpr. The potential for rapid expansion of the biofuels industry is
explored using a system-dynamics simulation model named the Biomass
Scenario Model (BSM), emphasizing how policy incentives and
technological learning-by-doing can accelerate industry growth. The BSM
simulates major sectors of the biofuels industry, including feedstock
production and logistics, conversion, distribution, and end use, as well
as their interaction with one another. The model represents conversion
of biomass to biofuels as a set of technology pathways, each having
characteristics that include allowable feedstocks, capital and operating
costs, and allowable products. Simulations indicate that coordination of
investments---with respect to timing, pathway, and target sector within
the biofuels industry---is needed to accelerate learning-by-doing and
most effectively expand biofuels production to meet Renewable Fuel
Standards (RFS). Metrics of effectiveness include timing and magnitude
of increased production, incentive cost and cost-effectiveness, and
avoidance of windfall profits. Due to risks and uncertainties,
investment costs and optimal investment targets---such as relative value
of investment in more-mature versus less-mature pathways---can be
explored through scenarios but not predicted with precision. During
rapid growth, dynamic competition intensifies, including competition for
cellulosic feedstocks and ethanol market shares. Rapid growth in ethanol
production occurs in simulations that allow higher blending proportions
of ethanol in gasoline-fueled vehicles, even up to RFS-targeted volumes
of biofuel.

\subsection{Journals}\label{journals}

Y. Alhassid and B. Bush, ``Stochastic approach to giant dipole
resonances in hot rotating nuclei,'' Phys. Rev.~Lett., vol.~63, no. 22,
pp.~2452--2455.
\textless{}http://link.aps.org/doi/10.1103/PhysRevLett.63.2452\textgreater{}A
stochastic macroscopic approach to giant dipole resonances (GDR's) in
hot rotating nuclei is presented. In the adiabatic limit the theory
reduces exactly to a previous adiabatic model where the unitary
invariant metric is used to calculate equilibrium averages. Nonadiabatic
effects cause changes in the GDR cross section and motional narrowing.
Comparisons with experiments where deviations from the adiabatic limit
are substantial are shown and can be used to determine the damping of
the quadrupole motion at finite temperature.

Y. Alhassid and B. Bush, ``Effects of thermal fluctuations on giant
dipole resonances in hot rotating nuclei,'' Nuclear Physics A, vol.~509,
no. 3, pp.~461--498.
\textless{}http://www.sciencedirect.com/science/article/pii/0375947490900873\textgreater{}We
present a macroscopic approach to giant dipole resonance (GDR) in highly
excited nuclei, using a unified description of quadrupole shape thermal
fluctuations. With only two free parameters, which are fixed by the
zero-temperature nuclear properties, the-model reproduces well
experimental GDR cross sections in the 100 ≤ A ≤ 170 mass range for both
spherical and deformed nuclei. We also investigate the cross-section
systematics as a function of both temperature and angular velocity and
the sensitivity of the GDR peak to the nuclear shape. We conclude that
at low temperatures (T ≈ 1 MeVfs) the GDR cross section is sensitive to
changes in the nuclear energy surface. Higher-temperature (T ≳ 2 MeV)
cross sections are dominated by large fluctuations (triaxial in
particular) and are much less sensitive to the equilibrium shape.

Y. Alhassid and B. Bush, ``Orientation fluctuations and the angular
distribution of the giant-dipole-resonance γ rays in hot rotating
nuclei,'' Phys. Rev.~Lett., vol.~65, no. 20, pp.~2527--2530.
\textless{}http://link.aps.org/doi/10.1103/PhysRevLett.65.2527\textgreater{}A
recent macroscopic approach to the giant dipole resonances in hot
rotating nuclei is extended to include the angular distributions of the
γ rays emitted in the resonance decay. It provides a uniform description
of thermal fluctuations in all quadrupole shape degrees of freedom
within the framework of the Landau theory. In particular, the inclusion
of fluctuations in the nuclear orientation with respect to the rotation
axis is crucial in reproducing the observed attenuation of the angular
anisotropy. The theory is applied to recent precision measurements in
90Zr and 92Mo and is the first to reproduce well both the observed
giant-dipole-resonance cross sections and the angular anisotropies.

Y. Alhassid and B. Bush, ``Time-dependent shape fluctuations and the
giant dipole resonance in hot nuclei: Realistic calculations,'' Nuclear
Physics A, vol.~514, no. 3, pp.~434--460.
\textless{}http://www.sciencedirect.com/science/article/pii/037594749090151B\textgreater{}The
effects of time-dependent shape fluctuations on the giant dipole
resonance (GDR) in hot rotating nuclei are investigated. Using the
framework of the Landau theory of shape transitions we develop a
realistic macroscopic stochastic model to describe the quadrupole
time-dependent shape fluctuations and their coupling to the dipole
degrees of freedom. In the adiabatic limit the theory reduces to a
previous adiabatic theory of static fluctuations in which the GDR cross
section is calculated by averaging over the equilibrium distribution
with the unitary invariant metric. Nonadiabatic effects are investigated
in this model and found to cause structural changes in the resonance
cross section and motional narrowing. Comparisons with experimental data
are made and deviations from the adiabatic calculations can be
explained. In these cases it is possible to determine from the data the
damping of the quadrupole motion at finite temperature.

Y. Alhassid and B. Bush, ``Time-dependent fluctuations and the giant
dipole resonance in hot nuclei: Solvable models,'' Nuclear Physics A,
vol.~531, no. 1, pp.~1--26.
\textless{}http://www.sciencedirect.com/science/article/pii/037594749190565N\textgreater{}A
recent macroscopic theory of time-dependent shape fluctuations in hot
nuclei and their effects on the giant dipole resonance is investigated
in the context of solvable models with one quadrupole shape degree of
freedom. Using the framework of the Landau theory of shape transitions,
both the quadrupole shape and the giant dipole degrees of freedom are
described by a coupled set of stochastic equations. Two solvable models
for which the dipole correlation function is found in closed form are
discussed; one for a spherical nucleus and one for a deformed nucleus.
The adiabatic and sudden limits of the models are examined. The latter
limit is shown to produce a phenomenon known as motional narrowing. For
the more general cases we introduce Monte Carlo techniques and test them
against the solvable models.

Y. Alhassid and B. Bush, ``Effects of orientation fluctuations on the
angular distribution of the giant dipole resonance γ-rays in hot
rotating nuclei,'' Nuclear Physics A, vol.~531, no. 1, pp.~39--62.
\textless{}http://www.sciencedirect.com/science/article/pii/037594749190567P\textgreater{}The
macroscopic approach to the GDR in hot rotating nuclei is extended to
include the angular distribution of the emitted GDR γ-rays. The effects
of thermal shape fluctuation and in particular fluctuations in the
nuclear orientation with respect to the rotation axis, are discussed in
the framework of the Landau theory. It is found that while orientation
fluctuations have negligible effects on the GDR cross section, they
cause significant attenuation in the angular anisotropy parameter a2
which offsets the a2 enhancement due to intrinsic shape fluctuations. It
is shown that this fluctuation theory is successful in reproducing both
the observed cross section and a2 in highly excited 90Zr and 92Mo
compound nuclei. The non-adiabatic effects on a2 are studied in terms of
a time-dependent model for the quadrupole shape fluctuations.

Y. Alhassid and B. Bush, ``The systematics of the Landau theory of hot
rotating nuclei,'' Nuclear Physics A, vol.~549, no. 1, pp.~12--42.
\textless{}http://www.sciencedirect.com/science/article/pii/037594749290065R\textgreater{}The
Landau theory of hot rotating nuclei, which was recently introduced to
explain the universal features of the shape transitions, is shown to
describe well many nuclei at moderate temperatures (T ≳ 1 MeV) and spin.
The Landau parameters are extracted from microscopic calculations. Their
systematics as a function of temperature and neutron numbers is
demonstrated for the neodynium isotopes with even number of neutrons. An
extended Landau theory is introduced to describe better nuclei at lower
temperatures and /or higher spins.

Y. Alhassid and B. Bush, ``Nuclear level densities in the static-path
approximation: (I). A solvable model,'' Nuclear Physics A, vol.~549, no.
1, pp.~43--58.
\textless{}http://www.sciencedirect.com/science/article/pii/037594749290066S\textgreater{}We
investigate the static-path approximation (SPA) and mean-field
approximation (MFA) for the level density within a solvable SU(2) model.
Comparing the SPA level density to the MFA one, we find an enhancement
with a great sensitivity to the interaction strength, in agreement with
exact analytic results. This enhancement compensates for a corresponding
suppression which occurs at negative temperatures. The saddle-point
approximation used in converting the partition function to the level
density works well at all but low energies.

Y. Alhassid and B. Bush, ``Nuclear level densities in the static-path
approximation: (II). Spin dependence,'' Nuclear Physics A, vol.~565, no.
2, pp.~399--426.
\textless{}http://www.sciencedirect.com/science/article/pii/037594749390218M\textgreater{}The
static-path approximation (SPA) and mean-field approximation (MFA) for
the partition function and level density are investigated with the
inclusion of spin. The methods are studied within a solvable model, the
nuclear SU(3) Elliot model. The SPA partition function is enhanced
compared with the MFA partition function and is in good agreement with
the exact result at all angular velocities (or spins) and at all but low
temperatures. The error made in the SPA as well as in the saddle-point
approximation used in the conversion from angular velocity to spin is
only weakly dependent on the spin and is small at not too low
temperatures (or excitation energies).

Y. Alhassid, B. Bush, and S. Levit, ``Landau theory of shapes, shape
fluctuations and giant dipole resonances in hot nuclei,'' Nuclear
Physics A, vol.~482, no. 1--2, pp.~57--64.
\textless{}http://www.sciencedirect.com/science/article/pii/0375947488905751\textgreater{}Universal
features of evolution of the equilibrium nuclear shapes with temperature
and angular momentum are predicted by the Landau theory of nuclear shape
transitions. The general dependence of the nuclear free energy on the
deformation given by this theory also provides a unified description of
thermal fluctuations of all quadrupole degrees of freedom. Using this
unified theory we calculate the giant dipole absorption by hot rotating
nuclei and investigate its systematics as a function of nuclear spin and
temperature. Direct comparison with experimental data is presented.

Y. Alhassid, B. Bush, and S. Levit, ``Thermal Shape Fluctuations, Landau
Theory, and Giant Dipole Resonances in Hot Rotating Nuclei,'' Phys.
Rev.~Lett., vol.~61, no. 17, pp.~1926--1929.
\textless{}http://link.aps.org/doi/10.1103/PhysRevLett.61.1926\textgreater{}A
macroscopic approach to giant dipole resonances (GDR's) in hot rotating
nuclei is presented. It is based on the Landau theory of nuclear shape
transitions and provides a unified description of thermal fluctuations
in all quadrupole shape degrees of freedom. With all parameters fixed by
the zero-temperature nuclear properties the theory shows a very good
agreement with existing GDR measurements in hot nuclei. The sensitivity
of the GDR peak to the shape of hot nuclei is critically examined.
Low-temperature experimental results in Er show clear evidence for
changes in the nuclear energy surface, while higher-temperature results
are dominated by the fluctuations.

M. Blue and B. W. Bush, ``Information content in the Nagel-Schreckenberg
cellular automaton traffic model,'' Phys. Rev.~E, vol.~67, no. 4,
p.~047103.
\textless{}http://link.aps.org/doi/10.1103/PhysRevE.67.047103\textgreater{}We
estimate the set dimension and find bounds for the set entropy of a
cellular automaton model for single lane traffic. Set dimension and set
entropy, which are measures of the information content per cell, are
related to the fractal nature of the automaton {[}S. Wolfram, Physica D
10, 1 (1989); Theory and Application of Cellular Automata, edited by S.
Wolfram (World Scientific, Philadelphia, 1986){]} and have practical
implications for data compression. For models with maximum speed vmax,
the set dimension is approximately log(vmax+2)2.5, which is close to one
bit per cell regardless of the maximum speed. For a typical maximum
speed of five cells per time step, the dimension is approximately 0.47.

M. Blue, B. Bush, and J. Puckett, ``Unified approach to fuzzy graph
problems,'' Fuzzy Sets and Systems, vol.~125, no. 3, pp.~355--368.
\textless{}http://www.sciencedirect.com/science/article/pii/S0165011401000112\textgreater{}We
present a taxonomy of fuzzy graphs that treats fuzziness in vertex
existence, edge existence, edge connectivity, and edge weight. Within
that framework, we formulate some standard graph-theoretic problems
(shortest paths and minimum cut) for fuzzy graphs using a unified
approach distinguished by its uniform application of guiding principles
such as the construction of membership grades via the ranking of fuzzy
numbers, the preservation of membership grade normalization, and the
``collapsing'' of fuzzy sets of graphs into fuzzy graphs. Finally, we
provide algorithmic solutions to these problems, with examples.

B. Bush and Y. Alhassid, ``On the width of the giant dipole resonance in
deformed nuclei,'' Nuclear Physics A, vol.~531, no. 1, pp.~27--38.
\textless{}http://www.sciencedirect.com/science/article/pii/037594749190566O\textgreater{}Applying
surface dissipation models to the Goldhaber-Teller model, we calculate
the dependence of the giant dipole resonance (GDR) width on the nuclear
quadrupole deformation. When expressed in units of the spherical width,
this width reduces to a purely geometrical elliptic integral. It is
shown to be very well approximated by the empirical power law with an
exponent of 1.6. This approach utilizes no free parameters and
reproduces the experimentally observed width dependence for GDR's built
on the ground state of heavy nuclei. The formula derived here plays an
important role in a recently developed macroscopic approach to the GDR
in hot rotating nuclei.

B. Bush and J. Nix, ``Classical Hadrodynamics: Foundations of the
Theory,'' Annals of Physics, vol.~227, no. 1, pp.~97--150.
\textless{}http://www.sciencedirect.com/science/article/pii/S0003491683710778\textgreater{}We
derive and discuss the classical relativistic equations of motion for an
action corresponding to extended nucleons interacting with massive,
neutral scalar and vector meson fields. This theory, which we call
classical hadrodynamics, is the classical analogue of the quantum
hadrodynamics of Serot and Walecka but without the assumptions of the
mean-field approximation and of point nucleons. The theory is manifestly
covariant and allows for non-equilibrium phenomena, interactions among
all nucleons, and meson production when used for applications such as
relativistic heavy-ion collisions. We review the history of classical
meson field theory, with special emphasis on issues related to
self-interaction, preacceleration, runaway solutions, and finite-size
effects. Sample calculations are presented for nucleon-nucleon
collisions at plab = 200 GeV/c, where we find that the theory provides a
physically reasonable description of gross features assaciated with the
dominating soft reactions. The equations of motion are practical to
solve numerically for ultrarelativistic heavy-ion collisions.

B. W. Bush and J. Nix, ``Classical hadrodynamics: application to soft
nucleon-nucleon collisions,'' Nuclear Physics A, vol.~560, no. 1,
pp.~586--602.
\textless{}http://www.sciencedirect.com/science/article/pii/037594749390116F\textgreater{}We
present results for soft nucleon-nucleon collisions at Plab = 14.6, 30,
60, 100 and 200 GeV/c calculated on the basis of classical hadrodynamics
for extended nucleons. This theory, which corresponds to nucleons of
finite size interacting with massive neutral scalar and vector meson
fields, is the classical analogue of the quantum hadrodynamics of Serot
and Walecka but without the assumptions of the mean-field approximation
and of point nucleons. The theory is manifestly Lorentz-covariant and
automatically includes space-time nonlocality and retardation,
nonequilibrium phenomena, interactions among all nucleons and particle
production when used for applications such as relativistic heavy-ion
collisions. We briefly review the history of classical meson-field
theory and present our classical relativistic equations of motion, which
are solved to yield such physically observable quantities as scattering
angle, transverse momentum, radiated energy and rapidity. We find that
the theory provides a physically reasonable description of gross
features associated with the soft reactions that dominate
nucleon-nucleon collisions. The equations of motion are practical to
solve numerically for relativistic heavy-ion collisions.

B. W. Bush and J. Nix, ``Classical hadrodynamics: A new approach to
ultrarelativistic heavy-ion collisions,'' Nuclear Physics A, vol.~583,
no. 0, pp.~705--710.
\textless{}http://www.sciencedirect.com/science/article/pii/037594749400748C\textgreater{}We
discuss a new approach to ultrarelativistic heavy-ion collisions based
on classical hadrodynamics for extended nucleons, corresponding to
nucleons of finite size interacting with massive meson fields. This new
theory provides a natural covariant microscopic approach that includes
automatically spacetime nonlocality and retardation, nonequilibrium
phenomena, interactions among all nucleons and particle production. In
the current version of our theory, we consider N extended unexcited
nucleons interacting with massive neutral scalar (σ) and neutral vector
(ω) meson fields. The resulting classical relativistic many-body
equations of motion are solved numerically without further approximation
for soft nucleon-nucleon collisions at plab = 14.6, 30, 60, 100 and 200
GeV/c to yield the transverse momentum imparted to the nucleons. For the
future development of the theory, the isovector pseudoscalar (π+, π−,
π0), isovector scalar (δ+, δ−, δ0), isovector vector (ϱ+, ϱ−, ϱ0) and
neutral pseudoscalar (η) meson fields that are known to be important
from nucleon-nucleon scattering experiments should be incorporated. In
addition, the effects of quantum uncertainty on the equations of motion
should be included by use of techniques analogous to those used by Moniz
and Sharp for nonrelativistic quantum electrodynamics.

B. W. BUSH and R. D. SMALL, ``A Note on the Ignition of Vegetation by
Nuclear Weapons,'' Combustion Science and Technology, vol.~52, no. 1-3,
pp.~25--38.
\textless{}http://www.tandfonline.com/doi/abs/10.1080/00102208708952566\textgreater{}Abstract
Smoke produced from the ignition and burning of live vegetation by
nuclear explosions has been suggested as a major contributor to a
possible nuclear winter. In this paper, we consider the mechanics of
live vegetation ignition by a finite-radius nuclear fireball. For
specified plant properties, the amount of fireball radiation absorbed by
a plant community is calculated as a function of depth into the stand
and range from the fireball. The spectral regions of plant energy
absorption and the overlap with the emitted fireball thermal spectra are
discussed. A simple model for the plant response to the imposed thermal
load is developed. First, the temperature is raised; the change depends
on the leaf structure, moisture content, and plant canopy. Subsequent
energy deposition desiccates the plant and finally raises its
temperature to the threshold ignition limit. Results show the
development of a variable depth ignition zone. Close to the fireball,
ignition of the entire plant occurs. At greater distances (several
fireball radii) portions of the plant are only partially desiccated, and
sustained burning is less probable. Far from the burst, the top of the
stand is weakly heated, and only a small transient temperature change
results. An estimate of the smoke produced by an exchange involving the
U.S. missile fields shows that the burning of live vegetation only
slightly increases the total nonurban smoke production.

B. W. Bush, G. F. Bertsch, and B. A. Brown, ``Shape diffusion in the
shell model,'' Phys. Rev.~C, vol.~45, no. 4, pp.~1709--1719.
\textless{}http://link.aps.org/doi/10.1103/PhysRevC.45.1709\textgreater{}The
diffusion coefficient for quadrupolar shape changes is derived in a
model based on the mixing of static Hartree-Fock configurations by the
residual interaction. The model correctly predicts the width of
single-particle configurations. We find a diffusion rate depending on
temperature as T3, consistent with at least one other theoretical
estimate. However, our diffusion rate is an order of magnitude lower
than two values extracted from data.

B. Bush, G. Anno, R. McCoy, R. Gaj, and R. D. Small, ``Fuel loads in
U.S. Cities,'' Fire Technology, vol.~27, no. 1, pp.~5--32.
\textless{}http://www.springerlink.com/content/k01511qw16kw1462/abstract/\textgreater{}Sources
of burnable material within U.S. cities are analyzed. Based on a
detailed evaluation of construction practices, storage of burnable
contents, building function and layout, and density of buildings in city
districts, we derive urban fuel load densities in terms of land use type
and geographic location. Residential building fuel loads vary regionally
from 123 to 150 kg m -2 ; non-residential building classes have loads
from 39 to 273 kg m -2 . The results indicate that average U.S. urban
area fuel loads range from 14 to 21 kg m -2 .

C. M. Clark, Y. Lin, B. G. Bierwagen, L. M. Eaton, M. H. Langholtz, P.
E. Morefield, C. E. Ridley, L. Vimmerstedt, S. Peterson, and B. W. Bush,
``Growing a sustainable biofuels industry: economics, environmental
considerations, and the role of the Conservation Reserve Program,''
Environ. Res. Lett., vol.~8, no. 2, p.~025016.
\textless{}http://iopscience.iop.org/1748-9326/8/2/025016\textgreater{}Biofuels
are expected to be a major contributor to renewable energy in the coming
decades under the Renewable Fuel Standard (RFS). These fuels have many
attractive properties including the promotion of energy independence,
rural development, and the reduction of national carbon emissions.
However, several unresolved environmental and economic concerns remain.
Environmentally, much of the biomass is expected to come from
agricultural expansion and/or intensification, which may greatly affect
the net environmental impact, and economically, the lack of a developed
infrastructure and bottlenecks along the supply chain may affect the
industry's economic vitality. The approximately 30 million acres (12
million hectares) under the Conservation Reserve Program (CRP) represent
one land base for possible expansion. Here, we examine the potential
role of the CRP in biofuels industry development, by (1) assessing the
range of environmental effects on six end points of concern, and (2)
simulating differences in potential industry growth nationally using a
systems dynamics model. The model examines seven land-use scenarios
(various percentages of CRP cultivation for biofuel) and five economic
scenarios (subsidy schemes) to explore the benefits of using the CRP.
The environmental assessment revealed wide variation in potential
impacts. Lignocellulosic feedstocks had the greatest potential to
improve the environmental condition relative to row crops, but the most
plausible impacts were considered to be neutral or slightly negative.
Model simulations revealed that industry growth was much more sensitive
to economic scenarios than land-use scenarios---similar volumes of
biofuels could be produced with no CRP as with 100\% utilization. The
range of responses to economic policy was substantial, including
long-term market stagnation at current levels of first-generation
biofuels under minimal policy intervention, or RFS-scale quantities of
biofuels if policy or market conditions were more favorable. In total,
the combination of the environmental assessment and the supply chain
model suggests that large-scale conversion of the CRP to row crops would
likely incur a significant environmental cost, without a concomitant
benefit in terms of biofuel production.

J. M. Fair, D. R. Powell, R. J. LeClaire, L. M. Moore, M. L. Wilson, L.
R. Dauelsberg, M. E. Samsa, S. M. DeLand, G. Hirsch, and B. W. Bush,
``Measuring the uncertainties of pandemic influenza,'' International
Journal of Risk Assessment and Management, vol.~16, no. 1, pp.~1--27.
\textless{}http://dx.doi.org/10.1504/IJRAM.2012.047550\textgreater{}It
has become critical to assess the potential range of consequences of a
pandemic influenza outbreak given the uncertainty about its disease
characteristics while investigating risks and mitigation strategies of
vaccines, antivirals, and social distancing measures. Here, we use a
simulation model and rigorous experimental design with sensitivity
analysis that incorporates uncertainty in the pathogen behaviour and
epidemic response to show the extreme variation in the consequences of a
potential pandemic outbreak in the USA. Using sensitivity analysis we
found the most important disease characteristics are the fraction of the
transmission that occur prior to symptoms, the reproductive number, and
the length of each disease stage. Using data from the historical
pandemics and for potential viral evolution, we show that response
planning may underestimate the pandemic consequences by a factor of two
or more.

S.-J. Lee, B. Bush, and R. George, ``Analytic science for geospatial and
temporal variability in renewable energy: A case study in estimating
photovoltaic output in Arizona,'' Solar Energy, vol.~85, no. 9,
pp.~1945--1956.
\textless{}http://www.sciencedirect.com/science/article/pii/S0038092X11001745\textgreater{}To
assess the electric power grid environment under the high penetration of
photovoltaic (PV) generation, it is important to construct an accurate
representation of PV power output for any location in the southwestern
United States at resolutions down to 10-min time steps. Existing
analyses, however, typically depend on sparsely spaced measurements and
often include modeled data as a basis for extrapolation.
Consequentially, analysts have been confronted with inaccurate analytic
outcomes due to both the quality of the modeled data and the
approximations introduced when combining data with differing space/time
attributes and resolutions. This study proposes an accurate methodology
for 10-min PV estimation based on the self-consistent combination of
data with disparate spatial and temporal characteristics. Our Type I
estimation uses the nearby locations of temporally detailed PV
measurements, whereas our Type II estimation goes beyond the spatial
range of the measured PV incorporating alternative data set(s) for areas
with no PV measurements; those alternative data sets consist of: (1)
modeled PV output and secondary cloud cover information around
space/time estimation points, and (2) their associated uncertainty. The
Type I estimation identifies a spatial range from existing PV sites
(30--40 km), which is used to estimate accurately 10-min PV output
performance. Beyond that spatial range, the data-quality-control
estimation (Type II) demonstrates increasing improvement over the Type I
estimation that does not assimilate the uncertainty of data sources. The
methodology developed herein can assist the evaluation of the impact of
PV generation on the electric power grid, quantify the value of measured
data, and optimize the placement of new measurement sites.

R. D. Small and B. W. Bush, ``Smoke Production from Multiple Nuclear
Explosions in Nonurban Areas,'' Science, vol.~229, no. 4712,
pp.~465--469.
\textless{}http://www.sciencemag.org/content/229/4712/465\textgreater{}The
amount of smoke that may be produced by wildland or rural fires as a
consequence of a large-scale nuclear exchange is estimated. The
calculation is based on a compilation of rural military facilities,
identified from a wide variety of unclassified sources, together with
data on their geographic positions, surrounding vegetation (fuel), and
weather conditions. The ignition area (corrected for fuel moisture) and
the amount of fire spread are used to calculate the smoke production.
The results show a substantially lower estimated smoke production (from
wildland fires) than in earlier ``nuclear winter'' studies. The amount
varies seasonally and at its peak is less by an order of magnitude than
the estimated threshold level necessary for a major attenuation of solar
radiation.

R. D. Small, B. W. Bush, and M. A. Dore, ``Initial Smoke Distribution
for Nuclear Winter Calculations,'' Aerosol Science and Technology,
vol.~10, no. 1, pp.~37--50.
\textless{}http://www.tandfonline.com/doi/abs/10.1080/02786828908959219\textgreater{}Mappings
showing the initial distribution of smoke from a 3000 MT strike against
over 4000 targets in the United States are presented. An attack of this
magnitude would attempt to deprive the United States of all military
capabilities and to destroy its industrial capacity. Most urban areas
would be affected and damage to the economic base would be substantial.
Smoke distributions are derived for global climate model computation
grids. Such distributions represent part of the initial conditions for a
simulation of climate modification. A much finer grid (2 × 1.5 degree)
mapping is also given. In the latter, mountain systems are resolved, and
the possible influence of topography on smoke movement is discussed.
Injection profiles determined from large area fire calculations show
that the initial distribution and smoke mass centroid depend on the
burning rate and fuel loading. Low fuel loadings or long burn times
indicate a constant mixing ratio injection with a fairly low altitude
centroid. A two-level constant mass density profile with a
midtroposphere centroid is more appropriate for cities that burn rapidly
or have higher combustible loadings. Wind patterns at a typical
injection height indicate the effect of a nonuniform source on the
initial global spread of smoke.

K. L. Summers, T. P. Caudell, K. Berkbigler, B. Bush, K. Davis, and S.
Smith, ``Graph visualization for the analysis of the structure and
dynamics of extreme-scale supercomputers,'' Information Visualization,
vol.~3, no. 3, p.~209--222.
\textless{}http://dx.doi.org/10.1057/palgrave.ivs.9500079\textgreater{}We
are exploring the development and application of information
visualization techniques for the analysis of new massively parallel
supercomputer architectures. Modern supercomputers typically comprise
very large clusters of commodity SMPs interconnected by possibly dense
and often non-standard networks. The scale, complexity, and inherent
non-locality of the structure and dynamics of this hardware, and the
operating systems and applications distributed over them, challenge
traditional analysis methods. As part of the á la carte (A Los Alamos
Computer Architecture Toolkit for Extreme-Scale Architecture Simulation)
team at Los Alamos National Laboratory, who are simulating these new
architectures, we are exploring advanced visualization techniques and
creating tools to enhance analysis of these simulations with intuitive
three-dimensional representations and interfaces. This work complements
existing and emerging algorithmic analysis tools. In this paper, we give
background on the problem domain, a description of a prototypical
computer architecture of interest (on the order of 10,000 processors
connected by a quaternary fat-tree communications network), and a
presentation of three classes of visualizations that clearly display the
switching fabric and the flow of information in the interconnecting
network.

J. van Schagen, Y. Alhassid, J. Bacelar, B. Bush, M. Harakeh, W.
Hesselink, H. Hofmann, N. Kalantar-Nayestanaki, R. Noorman, A. Plompen,
A. Stolk, Z. Sujkowski, and A. van der Woude, ``GDR dissipation and
nuclear shape in hot fast-rotating Dy nuclei,'' Physics Letters B,
vol.~308, no. 3--4, pp.~231--236.
\textless{}http://www.sciencedirect.com/science/article/pii/037026939391277T\textgreater{}The
statistical γ-ray decay of the GDR built on excited states in Dy nuclei
has been investigated for selected domains of angular momentum up to
about 70ħ and temperatures in the range 1--2 MeV. The GDR strength
distribution extracted from the data indicate large average nuclear
deformations (β ∼ 0.35) at high angular momentum and average
temperatures T ⩾ 1.5 MeV. The experimental observation is supported by
results from calculations in which thermal shape fluctuations are taken
into account around an oblate equilibrium deformation βeq. Although this
equilibrium deformation increases with angular momentum, the
calculations show rather large and constant average deformations 〈β
∼0.35.

J. van Schagen, Y. Alhassid, J. Bacelar, B. Bush, M. Haraken, W.
Hesselink, H. Hofmann, N. Kalantar-Nayestanaki, R. Noorman, A. Plompen,
A. Stolk, Z. Sujkowski, and A. van der Woude, ``GDR γ-ray decay in
156Dy∗ from regions selected on temperature and angular momentum,''
Physics Letters B, vol.~343, no. 1--4, pp.~64--68.
\textless{}http://www.sciencedirect.com/science/article/pii/037026939401467Q\textgreater{}The
strength distribution of the GDR built on highly excited states in a
restricted temperature domain in 156Dy and 155Dy nuclei has been deduced
by subtraction of γ-ray spectra obtained for the decay of 154Dy∗ and
156Dy∗ from regions selected on angular momentum. The resulting
difference spectra have been analyzed within the statistical model. The
results show a large deformation (\textbar{}β\textbar{} ∼ 0.51±0.29 and
0.35±0.14) for the angular-momentum regions with 〈J〉 ∼ 32h̵ at T ≈
1.8±0.2 MeV and 〈J〉 ∼ 46h̵ at T ≈ 1.7±0.2 MeV, respectively, in
satisfactory agreement with calculations performed in the framework of
Landau theory of shape transitions and statistical fluctuations. The
deduced centroid energies are in agreement with the systematics of the
GDR built on the ground state. The width of the GDR shows a systematic
increase with increasing temperature.

L. J. Vimmerstedt, B. Bush, and S. Peterson, ``Ethanol Distribution,
Dispensing, and Use: Analysis of a Portion of the Biomass-to-Biofuels
Supply Chain Using System Dynamics,'' PLoS ONE, vol.~7, no. 5,
p.~e35082.
\textless{}http://dx.doi.org/10.1371/journal.pone.0035082\textgreater{}The
Energy Independence and Security Act of 2007 targets use of 36 billion
gallons of biofuels per year by 2022. Achieving this may require
substantial changes to current transportation fuel systems for
distribution, dispensing, and use in vehicles. The U.S. Department of
Energy and the National Renewable Energy Laboratory designed a system
dynamics approach to help focus government action by determining what
supply chain changes would have the greatest potential to accelerate
biofuels deployment. The National Renewable Energy Laboratory developed
the Biomass Scenario Model, a system dynamics model which represents the
primary system effects and dependencies in the biomass-to-biofuels
supply chain. The model provides a framework for developing scenarios
and conducting biofuels policy analysis. This paper focuses on the
downstream portion of the supply chain--represented in the distribution
logistics, dispensing station, and fuel utilization, and vehicle modules
of the Biomass Scenario Model. This model initially focused on ethanol,
but has since been expanded to include other biofuels. Some portions of
this system are represented dynamically with major interactions and
feedbacks, especially those related to a dispensing station owner's
decision whether to offer ethanol fuel and a consumer's choice whether
to purchase that fuel. Other portions of the system are modeled with
little or no dynamics; the vehicle choices of consumers are represented
as discrete scenarios. This paper explores conditions needed to sustain
an ethanol fuel market and identifies implications of these findings for
program and policy goals. A large, economically sustainable ethanol fuel
market (or other biofuel market) requires low end-user fuel price
relative to gasoline and sufficient producer payment, which are
difficult to achieve simultaneously. Other requirements (different for
ethanol vs.~other biofuel markets) include the need for infrastructure
for distribution and dispensing and widespread use of high ethanol
blends in flexible-fuel vehicles.

E. Warner, D. Inman, B. Kunstman, B. Bush, L. Vimmerstedt, S. Peterson,
J. Macknick, and Y. Zhang, ``Modeling biofuel expansion effects on land
use change dynamics,'' Environmental Research Letters, vol.~8, no. 1,
p.~015003.
\textless{}http://iopscience.iop.org/1748-9326/8/1/015003/\textgreater{}Increasing
demand for crop-based biofuels, in addition to other human drivers of
land use, induces direct and indirect land use changes (LUC). Our system
dynamics tool is intended to complement existing LUC modeling approaches
and to improve the understanding of global LUC drivers and dynamics by
allowing examination of global LUC under diverse scenarios and varying
model assumptions. We report on a small subset of such analyses. This
model provides insights into the drivers and dynamic interactions of LUC
(e.g., dietary choices and biofuel policy) and is not intended to assert
improvement in numerical results relative to other works. Demand for
food commodities are mostly met in high food and high crop-based biofuel
demand scenarios, but cropland must expand substantially. Meeting
roughly 25\% of global transportation fuel demand by 2050 with biofuels
requires \textgreater{}2 times the land used to meet food demands under
a presumed 40\% increase in per capita food demand. In comparison, the
high food demand scenario requires greater pastureland for meat
production, leading to larger overall expansion into forest and
grassland. Our results indicate that, in all scenarios, there is a
potential for supply shortfalls, and associated upward pressure on
prices, of food commodities requiring higher land use intensity (e.g.,
beef) which biofuels could exacerbate.

\subsection{Proceedings}\label{proceedings}

C. Barrett, B. Bush, S. Kopp, H. Mortveit, and C. Reidys, ``Sequential
dynamical systems and applications to simulations,'' in Simulation
Symposium, 2000. (SS 2000) Proceedings. 33rd Annual, Washington D.C.,
pp.~245--252. Computer simulations are extensively used for business and
science applications. However a simulation generically generates a
certain class of dynamical system whose properties are poorly
understood. We address some theoretical issues of computer simulations
and illustrate our concepts for the simulation of circular one-lane
traffic. We propose a certain class of discrete dynamical systems (SDS)
that captures key features of computer simulations and then show how SDS
techniques can be applied to a case of infrastructure simulations

R. Bent, T. Djidjeva, B. Hayes, J. Holland, H. Khalsa, S. Linger, M.
Mathis, S. Mniszewski, and B. Bush, ``Hydra: a service oriented
architecture for scientific simulation integration,'' in Proceedings of
the 2009 Spring Simulation Multiconference, p.~54.
\textless{}http://public.lanl.gov/rbent/hydra-with-cover.pdf\textgreater{}

K. Berkbigler, G. Booker, B. Bush, K. Davis, and N. Moss, ``Simulating
the quadrics interconnection network,'' in High Performance Computing
Symposium 2003, Advance Simulation Technologies Conference 2003,
Orlando, Florida. We outline à la carte, an approach for simulating
computing architectures applicable to extreme-scale systems (thousands
of processors) and to advanced, novel architectural configurations, and
describe in detail our simulation model of the Quadrics interconnection
network. Our component-based design allows for the seamless assembly of
architectures from representations of workload, processor, network
interface, switches, etc., with disparate resolutions and fidelities,
into an integrated simulation model. This accommodates different case
studies that may require different levels of fidelity in various parts
of a system. Simple ping timings can be modeled to approximately 100 ns.
We present results comparing the simulated versus actual execution time
of a 3D neutron transport application run on a machine with a Quadrics
network.

B. W. Bush and J. R. Nix, ``Classical hadrodynamics for extended
nucleons,'' in Proc. 8th Winter Workshop on Nuclear Dynamics, Jackson
Hole, Wyoming, p.~311--316.
\textless{}http://www.osti.gov/energycitations/servlets/purl/5692537-Wl0CTO/\textgreater{}We
discuss a new approach to relativistic nucleus-nucleus collisions based
on classical hadrodynamics for extended nucleons, corresponding to
nucleons of finite size interacting with massive meson fields. This
theory provides a natural covariant microscopic approach to relativistic
nucleus-nucleus collisions that includes automatically spacetime
nonlocality and retardation, nonequilibrium phenomena, interactions
among all nucleons, and particle production. Inclusion of the finite
nucleon size cures the difficulties with preacceleration and runaway
solutions that have plagued the classical theory of self-interacting
point particles.

B. W. Bush and J. R. Nix, ``Particle-production mechanism in
relativistic heavy-ion collisions,'' in Proc. 7th Int. Conf. on Nuclear
Reaction Mechanism, Varenna, Italy, p.~592.
\textless{}http://www.osti.gov/energycitations/servlets/purl/10162609-gv1Z8s/native/\textgreater{}We
discuss the production of particles in relativistic heavy-ion collisions
through the mechanism of massive bremsstrahlung, in which massive mesons
are emitted during rapid nucleon acceleration. This mechanism is
described within the framework of classical hadrodynamics for extended
nucleons, corresponding to nucleons of finite size interacting with
massive meson fields. This new theory provides a natural covariant
microscopic approach to relativistic heavy-ion collisions that includes
automatically spacetime nonlocality and retardation, nonequilibrium
phenomena, interactions among all nucleons, and particle production.
Inclusion of the finite nucleon size cures the difficulties with
preacceleration and runaway solutions that have plagued the classical
theory of self-interacting point particles. For the soft reactions that
dominate nucleon-nucleon collisions, a significant fraction of the
incident center-of-mass energy is radiated through massive
bremsstrahlung. In the present version of the theory, this radiated
energy is in the form of neutral scalar (σ) and neutral vector (ω)
mesons, which subsequently decay primarily into pions with some photons
also. Additional meson fields that are known to be important from
nucleon-nucleon scattering experiments should be incorporated in the
future, in which case the radiated energy would also contain isovector
pseudoscalar (π+, π--, π0), isovector scalar (δ+, δ--, δ0), isovector
vector (ρ+, ρ--, ρ0), and neutral pseudoscalar (η) mesons.

B. W. Bush, J. R. Nix, and A. J. Sierk, ``Spacetime nonlocality and
retardation in relativistic heavy-ion collisions,'' in Presented at the
7th Winter Workshop on Nuclear Dynamics, Key West, 27 Jan. - 2 Feb.
1991, Key West, Florida, vols. -1, p.~282--287.
\textless{}http://adsabs.harvard.edu/abs/1991nudy.workR\ldots{}.B\textgreater{}We
discuss the exact numerical solution of the classical relativistic
equations of motion for a Lagrangian corresponding to point nucleons
interacting with massive scalar and vector meson fields. The equations
of motion contain both external retarded Lorentz forces and
radiation-reaction forces; the latter involve nonlocal terms that depend
upon the past history of the nucleon in addition to terms analogous to
those of classical electrodynamics. The resulting microscopic many-body
approach to relativistic heavy-ion collisions is manifestly Lorentz
covariant and allows for nonequilibrium phenomena, interactions with
correlated clusters of nucleons, and particle production. For point
nucleons, the asymptotic behavior of nucleonic motion prior to the
collision is exponential, with a range in proper time of approximately
0.5 fm. However, this behavior is altered by the finite nucleon size,
whose effect we are currently incorporating into our equations of
motion. The spacetime nonlocality and retardation that will be present
in the solutions of these equations may be responsible for significant
collective effects in relativistic heavy-ion collisions.

B. W. Bush, J. R. Nix, and A. J. Sierk, ``Classical hadrodynamics
approach to ultrarelativistic heavy‐ion collisions,'' in AIP Conference
Proceedings, Tucson, Arizona, vol.~243, pp.~835--837.
\textless{}http://proceedings.aip.org/resource/2/apcpcs/243/1/835\_1?isAuthorized=no\textgreater{}We
discuss the exact solution of the classical relativistic equations of
motion for an action corresponding to nucleons interacting with massive
scalar and vector meson fields. This model−the classical analogue of the
quantum hadrodynamics of Serot and Walecka−provides a manifestly Lorentz
covariant approach to heavy‐ion collisions, allows for nonequilibrium
phenomena, interactions of correlated nucleon clusters, and particle
production, and is valid when interaction times are short. We present an
analysis of the nonlocality inherent in the model and discuss effects
arising from the finite size of a nucleon.

B. Bush, Dauelsberg, L., LeClaire, R., Powell, D., DeLand, S., and
Samsa, M., ``Critical infrastructure protection decision support system
(CIP/DSS) project overview,'' in Proceedings of the 2005 System Dynamics
Conference, Boston.
\textless{}http://www.systemdynamics.org/conferences/2005/proceed/papers/LECLA332.pdf\textgreater{}The
Critical Infrastructure Protection Decision Support System (CIP/DSS)
simulates the dynamics of individual infrastructures and couples
separate infrastructures to each other according to their
interdependencies. For example, repairing damage to the electric power
grid in a city requires transportation to failure sites and delivery of
parts, fuel for repair vehicles, telecommunications for problem
diagnosis and coordination of repairs, and the availability of labor.
The repair itself involves diagnosis, ordering parts, dispatching crews,
and performing work. The electric power grid responds to the initial
damage and to the completion of repairs with changes in its operating
characteristics. Dynamic processes like these are represented in the
CIP/DSS infrastructure sector simulations by differential equations,
discrete events, and codified rules of operation. Many of these
variables are output metrics estimating the human health, economic, or
environmental effects of disturbances to the infrastructures.

B. Bush, M. Duffy, D. Sandor, and S. Peterson, ``Using system dynamics
to model the transition to biofuels in the United States,'' presented at
the Third International Conference on System of Systems Engineering,
Monterey, California.
\textless{}http://www.nrel.gov/docs/fy08osti/43153.pdf\textgreater{}Transitioning
to a biofuels industry that is expected to displace about 30\% of
current U.S. gasoline consumption requires a robust biomass-to-biofuels
system-of-systems that operates in concert with the existing markets.
This paper discusses employing a system dynamics approach to investigate
potential market penetration scenarios for cellulosic ethanol and to
help government decision makers focus on areas with greatest potential.

B. Bush, M. Duffy, D. Sandor, and S. Peterson, ``Using system dynamics
to model the transition to biofuels in the United States,'' presented at
the SoSE '08. IEEE International Conference on System of Systems
Engineering, Singapore.
\textless{}http://www.nrel.gov/docs/fy08osti/43153.pdf\textgreater{}Today,
the U.S. consumes almost 21 million barrels of crude oil per day;
approximately 60\% of the U.S. demand is supplied by imports. The
transportation sector alone accounts for two-thirds of U.S. petroleum
use. Biofuels, liquid fuels produced from domestically-grown biomass,
have the potential to displace about 30\% of current U.S. gasoline
consumption. Transitioning to a biofuels industry on this scale will
require the creation of a robust biomass-to-biofuels system-of-systems
that operates in concert with the existing agriculture, forestry,
energy, and transportation markets. The U.S. Department of Energy is
employing a system dynamics approach to investigate potential market
penetration scenarios for cellulosic ethanol, and to aid decision makers
in focusing government actions on the areas with greatest potential to
accelerate the deployment of biofuels and ultimately reduce the
nationpsilas dependence on imported oil.

B. Bush, O. Sozinova, and M. Melaina, ``Optimal Regional Layout of
Least-Cost Hydrogen Infrastructure,'' in Proceedings of the 2010 NHA
Hydrogen Conference \& Expo, Washington, DC, vol.~28.

Bush, B. and Ivey, A., ``Numerical Hurricane Model Outputs for GIS-Based
Infrastructure Damage Estimation,'' in 2006 ESRI Federal User
Conference, Washington, D.C.
\textless{}http://proceedings.esri.com/library/userconf/feduc06/docs/hurricane\_damage\_model.pdf\textgreater{}The
wind and precipitation fields forecast by numerical weather prediction
(NWP) models, combined with the output of storm surge models, can
provide estimates of damage to infrastructures such as the electric
power grid several days before a hurricane makes landfall. Having the
hourly forecasts of grid-based meteorological fields imported into a GIS
enables an analyst to compute the cumulative effects over time of wind
and rain and, subsequently, to overlay these with storm surge,
elevation, and infrastructure data in order to categorize the forecast
exposure of facilities to extreme weather. Calibrated heuristic models
are then applied within the GIS to compute expected damage from the
forecasted exposure. We provide examples for hurricanes from the 2005
season in the Atlantic Ocean and Gulf of Mexico using the output of
several publicly available NWP model forecasts; similar methods apply to
other types of extreme weather such as ice storms.

Bush, B.W. and Nix, J.R., ``Calculations of Ultrarelativistic
Nucleus-Nucleus Collisions Based on Classical Hadrodynamics for Extended
Nucleons,'' in Contributed Papers and Abstracts, Quark Matter '91, Ninth
Int. Conf. on Ultra-Relativistic Nucleus-Nucleus Collisions, Gatlinburg,
Tennessee, 1991, Gatlinburg, Tennessee, p.~T98.

Bush, Brian W., Conrad, Stephen H., DeLand, Sharon M., Martínez-Moyano,
Ignacio J., Powell, Dennis R., and Zagonel, Aldo A., ``Working with
`living' models: Emergent methodological contributions from modeling for
critical infrastructure protection,'' in Proceedings of the 26 th
International Conference of the System Dynamics Society, Athens, Greece.
\textless{}http://www.google.com/url?sa=t\&rct=j\&q=\&esrc=s\&source=web\&cd=1\&cad=rja\&ved=0CCIQFjAA\&url=http\%3A\%2F\%2Fwww.systemdynamics.org\%2Fconferences\%2F2008\%2Fproceed\%2Fpapers\%2FZAGON282.pdf\&ei=NjxCUN7aN5P02wWy9oHQDA\&usg=AFQjCNGnahVTiffPW\_It2Ms6uesZDMeu9w\&sig2=6iJbhxQfFgcMplMZaFx1fw\textgreater{}Critical
infrastructures are increasingly automated and interdependent, subject
to possibly cascading vulnerabilities due to equipment failures, natural
disasters, and terrorist attacks. The government seeks to ensure that
disruptions are infrequent, brief, manageable, and cause the least harm
possible. The system dynamics (SD) approach is particularly promising as
a way to understand these complex systems, interactions, and issues.
Problems in critical infrastructure protection are being investigated
with a collection of SD models developed expressly for these concerns,
including agriculture models. This paper discusses the technical and
social modeling context that makes this SD modeling effort seem
uncommon. It involves a modular approach, a model-reassembling
technology, a formal process for testing and evaluation, and a social
process for managing the development and use of ``living'' models.

C. Unal, B. Bush, K. Werley, and P. Giguere, ``Modeling of
Interdependent Infrastructures,'' in Probabilistic safety assessment and
management (PSAM6) : proceedings of the 6th International Conference on
Probabilistic Safety Assessment and Management, 23-28 June 2002, San
Juan, Puerto Rico, USA, San Juan, Puerto Rico. An actor-based modeling
methodology is used to simulate interactions among interdependent
commercial infrastructures. The goal of this method is to capture the
complex, nonlinear, self-organizing, emergent, and sometimes chaotic
behaviors and interactions exemplified by complex systems, rather than
relying on traditional aggregate mathematical and simulation techniques.
A prototype model of four interdependent infrastructures was considered
as an example. The actor-based definitions of the electric-power
transmission line and natural-gas pipeline networks were developed to
realistically simulate the dynamic interactions within each of these
infrastructures and the interactions and interdependencies between these
two infrastructures. A three-dimensional representation of system
components and interconnectivity was developed. The visualization is an
interactive, three-dimensional, geographically based, ``layered'' view
of infrastructure interdependencies. It also links to a geographic
information system for data analysis. A unique iterative natural-gas
network solver algorithm was developed. Our assessment shows that a
hybrid approach using an actor-based definition of infrastructure
components in conjunction with iterative and commercial solvers has
great promise for addressing the operation of interdependent
infrastructures in a restructured and deregulated environment.

Dauelsberg, Lori R., Powell, Dennis R., LeClaire, Rene J., Bush, Brian
W., DeLand, Sharon M., and Samsa, Michael E., ``Critical Infrastructure
Protection Decision Support System Overview,'' in Risk Analysis for
Homeland Security and Defense Theory and Application, Santa Fe, New
Mexico.

Fernandez, S.J., Bush, B., Toole, G.L., Dauelsberg, L., Flaim, S.,
Thayer, G.R., and Ivey, A., ``Predicting Hurricane Impacts on the
Nation's Infrastructure: Lessons Learned from the 2005 Hurricane
Season,'' in Second International Conference on Global Warming, Santa
Fe, New Mexico. During the 2005 Hurricane season, many consequence
predictions were available to key Federal agencies from 36 to 96 hours
before each of the major hurricane US mainland landfalls. These key
forecasts included the location and intensity of the hurricane at
landfall, areas of significant damage to engineered infrastructure and
lifeline utilities, time estimates to restore critical infrastructure
services, and the conditions to be found on the ground as emergency and
relief crews enter the area. Both the Department of Energy through its
Visualization and Modeling Working Group and the Department of Homeland
Security provided early forecasts of potential damage to the regionally
critical infrastructures. These products communicated critical
information that assisted in the decision-making process for emergency
planning.

Fernandez, Steven, Bush, Brian, Toole, G. Loren, Dauelsberg, Lori, and
Flaim, Silvio, ``Lessons Learned from Infrastructure Impacts of
Hurricanes Dennis, Katrina, Rita and Wilma,'' in 60th Interdepartmental
Hurricane Conference, Mobile, Alabama.
\textless{}http://www.ofcm.gov/ihc06/Presentations/06\%20session6\%20Decision-making\%20Products/s6-01Fernandez.pdf\textgreater{}

Kathryn Berkbigler, Brian Bush, Kei Davis, Nicholas Moss, Steve Smith,
Thomas P. Caudell, Kenneth L. Summers, and Cheng Zhou, ``À la carte: A
Simulation Framework for Extreme-Scale Hardware Architectures,'' in MS
2003: IASTED International Conference on Modelling and Simulation, Palm
Springs, California. We outline à la carte, an approach for simulating
computing architectures applicable to extreme-scale systems (thousands
of processors) and to advanced, novel architectural configurations. Our
component-based design allows for the seamless assembly of architectures
from representations of workload, processor, network interface,
switches, etc., with disparate resolutions, into an integrated
simulation model. This accommodates different case studies that may
require different levels of fidelity in various parts of a system. The
current implementation includes low- and medium-fidelity models of the
network and low-fidelity and direct execution models of the workload. It
supports studies of both simulation performance and scaling, and the
properties of the simulated system themselves.

LeClaire, Rene, Dauelsberg, Lori, Bush, Brian, Beyeler, Walt, Conrad,
Stephen, and O'Reilly, Gerard, ``Infrastructure Interdependency
Consequence Analysis of a Telecommunications Disruption,'' in Working
Together: R\&D Partnerships in Homeland Security, Boston. The Critical
Infrastructure Protection Decision Support System (CIP/DSS) is being
developed by the Science and Technology Directorate of the Department of
Homeland Security to provide a risk-informed decision aid for the
evaluation of alternate protective measures and investment strategies in
support of critical infrastructure protection. In this paper, we will
describe the development of a suite of coupled infrastructure
consequence models and their application to the analysis of a disruption
in the telecommunications infrastructure. The model suite includes
models that operate at different geographic scales (metropolitan and
regional/national) to develop a better understanding of both local and
national-scale effects. We will present model results for a postulated
telecommunications disruption resulting in loss of capacity in three
different large metropolitan areas. The results include the interplay
between behavioral responses (i.e., increased demand due to a mass
calling event) and recovery and restoration within the
telecommunications infrastructure as well as impacts in other
infrastructures. A multiattribute utility theory based approach was used
to evaluate trade-offs between different protective measures. The
analysis benefited substantially from knowledge gained from
telecommunication industry models and provides an example of how
industry and the national laboratories can collaborate in addressing
homeland security issues.

LeClaire, Rene, Dauelsberg, Lori, Bush, Brian, Fair, J., Powell, D.,
Deland, S.M., Beyeler, W.E., Min, H., R. Raynor, M. E. Samsa, R.
Whitfield, and G. Hirsch, ``Critical Infrastructure Protection Decision
Support System: Evaluation of a Biological Scenario,'' in Working
Together: R\&D Partnerships in Homeland Security, Boston.

S. Peterson, E. Newes, D. Inman, L. Vimmerstedt, D. Hsu, C. Peck, D.
Stright, and B. Bush, ``An Overview of the Biomass Scenario Model,''
presented at the The 31st International Conference of the System
Dynamics Society, Cambridge, Massachusetts.
\textless{}http://www.systemdynamics.org/conferences/2013/proceed/papers/P1352.pdf\textgreater{}Biofuels
are promoted in the United States through aggressive legislation as one
part of an overall strategy to lessen dependence on imported energy as
well as to reduce the emissions of greenhouse gases. Meeting mandated
volumetric targets has prompted substantial funding for biofuels
research, much of it focused on producing ethanol and other fuel types
from biomass feedstocks. A variety of incentive programs (including
subsidies, fixed capital investment grants, loan guarantees, vehicle
choice credits, and aggressive corporate average fuel economy
standards)have been developed, but their short-and long-term
ramifications are not well known. This paper describes the Biomass
Scenario Model, a system dynamics model developed under the support of
the U.S. Department of Energy as the result of a multi-year project at
the National Renewable Energy Laboratory. The model represents multiple
pathways leading to the production of fuel ethanol as well as advanced
biofuels such as biomass-based gasoline, diesel, jet fuel, and butanol).
This paper details the BSM system dynamics architecture, the design of
the supporting database infrastructure, the associated scenario
libraries used in model runs, as well as key insights resulting from BSM
simulations and analyses.

Small, R. and Bush, B., ``Smoke Production for Nuclear Attack
Scenarios,'' in Proc. Smoke Obscuration Symposium XIV, Laurel, Maryland.

D. Thompson and B. W. Bush, ``Group Development Software for Vensim®,''
presented at the 24th International Conference of the System Dynamics
Society in Nijmegen, The Netherlands.
\textless{}http://www.systemdynamics.org/conferences/2005/proceed/papers/LECLA437.pdf\textgreater{}The
development of large software systems using systems dynamics languages
such as Vensim® has been hampered by the lack of means to develop
modules independently and subsequently link, integrate, or merge the
modules. Most modern software languages support such a capability. A
software tool to facilitate group development of systems dynamics code
and the development conventions to support the process has proved
successful in a large code development project. The tool, Conductor, is
generally applicable to other projects using Vensim®.

Thompson, D. and Bush, B., ``Software Practices Applied to Systems
Dynamics: Support for Large Scale Group Development,'' in 3rd
International Conference of the System Dynamics Society, Boston.
\textless{}http://www.systemdynamics.org/conferences/2005/proceed/papers/LECLA437.pdf\textgreater{}The
development of large software systems using systems dynamics languages
has been hampered by the lack of application of, and support for, modern
software techniques. Support is needed to handle the challenges of
modular system dynamics model development. These development challenges
include the handling of namespaces, linking separate modes, and
maintaining clean logical separations among components. Most modern
software patterns and languages support such a capability. This paper
presents an approach to group, large scale, system dynamics model
development that has proven valuable in our project. Our approach
included the creation of a software tool, called Conductor, to
facilitate our group development. The tool, Conductor, is generally
applicable to other projects using Vensim®.

Toole, L., Flaim, S., Fernandez, S.J., Bossert, J., Bush, B., and
Neenan, B., ``Effects of Climate Change on California Energy Security,''
in SSR 2006: International Symposium on Systems \& Human Science,
Vienna, Austria. Sound energy planning requires an understanding of the
primary external drivers that will impact national futures over the next
25 years. One of the principal impacts may be from climate
variability---induced by both natural cycles and anthropogenic forcing.
The Southwestern United States is a case in point. This semi-arid region
has seen its water and landscape resources threatened by extended
drought, and it will remain in a precarious position---projected to
increase its population by 50\% over the next 25 years and highly
susceptible to the vagaries of climate change. This study, performed at
Los Alamos National Laboratory, focused on climate change impacts that
may occur within the regional surrounding California prior to 2035.
Results from the Hadley CM-3 climate model were used as drivers for a
comprehensive electric energy model of the California regional network.
Approximately 57 Gigawatts (GWe) of new generation capacity will be
required starting in 2022 due only to population growth and economic
expansion. Additional generation capacity will be required starting in
2015 due to climate change to meet higher electricity demand. To pass
through or mitigate such effects to the consumer, Federal and state
government will require new regulatory and pricing policies. This paper
discusses the adaptive changes required to address regulatory, social
and physical security issues related to climate change.

D. Visarraga, T. N. McPherson, S. P. Linger, and B. Bush, ``Development
of a JAVA Based Water Distribution Simulation Capability for
Infrastructure Interdependency Analyses,'' in Impacts of Global Climate
Change, Anchorage, Alaska, pp.~1--8.
\textless{}http://ascelibrary.org/doi/abs/10.1061/40792\%28173\%2914\textgreater{}A
linear theory approach is applied to the hydraulic simulation of a water
distribution system within the Interdependent Energy Infrastructure
Simulation System (IEISS). IEISS is an actor-based infrastructure
modeling, simulation, and analysis tool designed to assist individuals
in analyzing and understanding interdependent energy infrastructures. In
particular, it has the ability to analyze and simulate the
interdependent electric power and natural gas infrastructures. The
ultimate goal for IEISS is a multi-infrastructure modeling framework
that can be used to analyze the complex, nonlinear interactions among
interdependent infrastructures including electric power, natural gas,
petroleum, water, and other network based infrastructures that is
scalable to multiple spatial (e.g., urban to regional) and temporal
resolutions. The actor-based infrastructure components were developed in
IEISS to realistically simulate the dynamic interactions within each of
the infrastructures, as well as, the interconnections between the
infrastructures. To enhance its capabilities, a generalized fluid
network will be added to the infrastructure framework, which will allow
for the analysis of specific fluid infrastructures (e.g., water,
petroleum, oil, etc.). In this research, we describe the extension of
IEISS to include water infrastructure. The resulting simulation
capability (i.e., IEISS Water) will allow the simulation of
transmission/distribution-level water systems in terms of infrastructure
specific vulnerabilities and interdependent infrastructure
vulnerabilities (e.g., power and water disruptions).

C. Zhou, K. L. Summers, and T. P. Caudell, ``Graph visualization for the
analysis of the structure and dynamics of extreme-scale
supercomputers,'' in Proceedings of the 2003 ACM symposium on Software
visualization, San Diego, California, p.~143--149.
\textless{}http://doi.acm.org/10.1145/774833.774854\textgreater{}We are
exploring the development and application of information visualization
techniques for the analysis of new massively parallel supercomputer
architectures. Modern supercomputers typically comprise very large
clusters of commodity SMPs interconnected by possibly dense and often
nonstandard networks. The scale, complexity, and inherent nonlocality of
the structure and dynamics of this hardware, and the systems and
applications distributed over it, challenge traditional analysis
methods. As part of the 'a la carte team at Los Alamos National
Laboratory, who are simulating these advanced architectures, we are
exploring advanced visualization techniques and creating tools to
provide intuitive exploration, discovery, and analysis of these
simulations. This work complements existing and emerging algorithmic
analysis tools. This paper gives background on the problem domain, a
description of a prototypical computer architecture of interest (on the
order of 10,000 processors connected by a quaternary fat-tree
communications network), and a presentation of two classes of
visualizations that clearly display the switch structure and the flow of
information in the interconnecting network.

\subsection{Posters}\label{posters}

A. Ivey, L. Toole, B. Bush, and S. Fernandez, ``Impacts of Hurricanes on
Critical Infrastructure,'' presented at the U.S. Department of Energy
GIS Expo, Washington, D.C.

B. Bush, L. Dauelsberg, A. Ivey, R. LeClaire, D. Powell, S. DeLand, and
M. Samsa, ``Critical Infrastructure Protection Decision Support System
(CIP/DSS) Project Overview,'' presented at the 3rd International
Conference of the System Dynamics Society, Boston. The Critical
Infrastructure Protection Decision Support System (CIP/DSS) simulates
the dynamics of individual infrastructures and couples separate
infrastructures to each other according to their interdependencies. For
example, repairing damage to the electric power grid in a city requires
transportation to failure sites and delivery of parts, fuel for repair
vehicles, telecommunications for problem diagnosis and coordination of
repairs, and the availability of labor. The repair itself involves
diagnosis, ordering parts, dispatching crews, and performing work. The
electric power grid responds to the initial damage and to the completion
of repairs with changes in its operating characteristics. Dynamic
processes like these are represented in the CIP/DSS infrastructure
sector simulations by differential equations, discrete events, and
codified rules of operation. Many of these variables are output metrics
estimating the human health, economic, or environmental effects of
disturbances to the infrastructures.

Brian Bush and Cetin Unal, ``Simulation of Interdependent
Infrastructures,'' presented at the Los Alamos National Laboratory
Science Day, Los Alamos, New Mexico.

Donatella Pasqualini, M. Witkowski, B. Bush, D. Powell, R. LeClaire, L.
Dauelsberg, A. Outkin, J. Fair, and P. Klare, ``System Dynamics Approach
for Critical Infrastructure and Decision Support,'' presented at the
2005 ESRI International User Conference, San Diego, California.

Francis Alexander, Kathryn Berkbigler, Graham Booker, Brian Bush, Thomas
Caudell, Kei Davis, Tim Eyring, Adolfy Hoisie, Donner Holten, Steve
Smith, and Kenneth Summers, ``Extreme-Scale Architecture Simulation,''
presented at the SC'2001 Research Poster Session, Denver, Colorado.

J. Hacker, Brian Bush, and Jennifer Boehnert, ``GIS-Based Weather
Warnings from a WRF Ensemble,'' presented at the 8th WRF Users'
Workshop, Boulder, Colorado.

Kriste Henson, Ed Van Eeckhout, and Brian Bush, ``Visualizing Simulation
Data for a Metropolitan Area,'' presented at the 1999 ESRI Conference,
San Diego, California.

Leslie Moore, Dennis Powell, Brian Bush, and Rene LeClaire, ``CIPDSS:
Critical Infrastructure Protection Decision Support System,'' presented
at the RSS 2006 Conference, Belfast.

Lori R. Dauelsberg, Dennis R. Powell, Rene J. LeClaire, Brian W. Bush,
Sharon M. DeLand, and Michael E. Samsa, ``An Overview of CIPDSS,''
presented at the Risk Analysis for Homeland Security and Defense Theory
and Application, Santa Fe, New Mexico.

Rene LeClaire, Lori Dauelsberg, Brian Bush, Walt Beyeler, Stephen
Conrad, Gerard O'Reilly, William Buehring, Ronald Whitfield, and Michael
Samsa, ``Infrastructure Interdependency Consequence Analysis of a
Telecommunications Disruption,'' presented at the Working Together: R\&D
Partnerships in Homeland Security, Boston.

\subsection{Book Chapters}\label{book-chapters}

Carlisle, N. and Bush, B., ``Closing the Planning Gap: Moving to
Renewable Communities,'' in 100\% Renewable: Energy Autonomy in Action,
Droege, P., Ed. London: Earthscan, pp.~263--288.
\textless{}http://nrelpubs.nrel.gov/Webtop/ws/nich/www/public/Record?rpp=50\&upp=0\&m=7\&w=NATIVE\%28\%27AUTHOR+ph+words+\%27\%27Bush\%27\%27\%27\%29\&order=native\%28\%27pubyear\%2FDescend\%27\%29\textgreater{}

E. Newes, D. Inman, and B. Bush, ``Understanding the Developing
Cellulosic Biofuels Industry through Dynamic Modeling,'' in Economic
Effects of Biofuel Production, M. A. dos Santos Bernardes, Ed. Rijeka,
Croatia: InTech, pp.~373--404.
\textless{}http://www.intechopen.com/books/economic-effects-of-biofuel-production/understanding-the-developing-cellulosic-biofuels-industry-through-dynamic-modeling\textgreater{}Biofuels
are promoted in the United States through aggressive legislation, as one
part of an overall strategy to lessen dependence on imported energy as
well as to reduce the emissions of greenhouse gases (Office of the
Biomass Program and Energy Efficiency and Renewable Energy, 2008). For
example, the Energy Independence and Security Act of 2007 (EISA)
mandates 36 billion gallons of renewable liquid transportation fuel in
the U.S. marketplace by the year 2022 (U.S. Government, 2007). Meeting
such large volumetric targets has prompted an unprecedented increase in
funding for biofuels research. Language in the EISA legislation limits
the amount of renewable fuel derived from starch-based feedstocks (which
are already established and feed the commercially viable ethanol
industry in the United States); therefore, much of the current research
is focused on producing ethanol---but from cellulosic feedstocks. These
feedstocks, such as agricultural and forestry residues, perennial
grasses, woody crops, and municipal solid wastes, are advantageous
because they do not necessarily compete directly with food, feed, and
fiber production and are envisaged to require fewer inputs (e.g., water,
nutrients, and land) as compared to corn and other commodity crops. In
order to help propel the biofuels industry in general and the cellulosic
ethanol industry in particular, the U.S. government has enacted
subsidies, fixed capital investment grants, loan guarantees, vehicle
choice credits, and aggressive corporate average fuel economy standards
as incentives. However, the effect of these policies on the cellulosic
ethanol industry over time is not well understood. Policies such as
those enacted in the United States, that are intended to incentivize the
industry and promote industrial expansion, can have profound long-term
effects on growth and industry takeoff as well as interact with other
policies in unforeseen ways (both negative and positive). Qualifying the
relative efficacies of incentive strategies could potentially lead to
faster industry growth as well as optimize the government's investment
in policies to promote renewable fuels. The purpose of this chapter is
to discuss a system dynamics model called the Biomass Scenario Model
(BSM), which is being developed by the U.S. Department of Energy as a
tool to better understand the interaction of complex policies and their
potential effects on the burgeoning cellulosic biofuels industry in the
United States. The model has also recently been expanded to include
advanced conversion technologies and biofuels (i.e., conversion pathways
that yield biomass-based gasoline, diesel, jet fuel, and butanol), but
we focus on cellulosic ethanol conversion pathways here. The BSM uses a
system dynamics modeling approach (Bush et al., 2008) built on the
STELLA software platform (isee systems, 2010) to model the entire
biomass-to-biofuels supply chain. Key components of the BSM are shown in
Figure 1. In addition to describing the underpinnings of this model, we
will share insights that have been gleaned from a myriad of scenario-
and policy-driven model runs. These insights will focus on how
roadblocks, bottlenecks, and incentives all work in concert to have
profound effects on the future of the industry.

\subsection{Presentations}\label{presentations}

B. Bush, ``Biomass-to-Bioenergy Supply-Chain Scenario Analysis,''
presented at the 2013 Bioenergy Technologies Office Analysis and
Sustainability Peer Review, Alexandria, Virginia.
\textless{}https://www2.eere.energy.gov/biomass/peer\_review2013/Portal/presenters/public/InsecureDownload.aspx?filename=Peer\_Review\_Analysis\_Bush\_4.pdf\textgreater{}The
Biomass Scenario Model (BSM) is a unique, carefully validated,
state-of-the-art third-generation model of the domestic biofuels supply
chain which explicitly focuses on policy issues and their potential side
effects. It integrates resource availability, behavior, policy, and
physical, technological, and economic constraints. The model uses a
system-dynamics simulation (not optimization) to model dynamic
interactions across the supply chain; the BSM tracks the deployment of
biofuels given technological development and the reaction of the
investment community to those technologies in the context of land
availability, the competing oil market, consumer demand for biofuels,
and government policies over time. It places a strong emphasis on the
behavior and decision-making of various economic agents among ten
geographic regions domestically. The BSM has been used to develop
insights into biofuels industry growth and market penetration,
particularly with respect to policies and incentives applicable to each
supply-chain element (volumetric, capital, operating subsidies; carbon
caps/taxes; R\&D investment; loan guarantees; tax credits); the model
treats the major infrastructure-compatible fuels such as biomass-based
gasoline, diesel, jet fuel, ethanol, and butanol. In general, scenario
analysis based on the BSM shows that the biofuels industry tends not to
rapidly thrive without significant external actions in the early years
of its evolution. An initial focus for jumpstarting the industry
typically has strongest results in the BSM in areas where effects of
intervention have been identified to be multiplicative: due to
industrial learning dynamics, support for the construction of biofuel
conversion facilities in the near future encourages the industry to
flourish. In general, we find that policies which are coordinated across
the whole supply chain have significant impact in fostering the growth
of the biofuels industry and that the production of tens of billions of
gallons of biofuels may occur under sufficiently favorable conditions.

B. Bush, ``Biomass Scenario Model (BSM) Development \& Analysis,''
presented at the 2011 Office of the Biomass Program Analysis and
Sustainability Activities Platform Peer Review, Annapolis, Maryland.
\textless{}http://www.obpreview2011.govtools.us/presenters/public/InsecureDownload.aspx?filename=WBS\_6.2.1.2d\_PNNL\%20Algae\%20Resource\%20Assessment\%20April\%204\%202011\%20DOE\_V8.pdf\textgreater{}The
Biomass Scenario Model (BSM) is a unique, carefully validated,
state-of-the-art third-generation model of the domestic biofuels supply
chain which explicitly focuses on policy issues and their potential side
effects. It integrates resource availability, behavior, policy, and
physical, technological, and economic constraints. The model uses a
system-dynamics simulation (not optimization) to model dynamic
interactions across the supply chain; the BSM tracks the deployment of
biofuels given technological development and the reaction of the
investment community to those technologies in the context of land
availability, the competing oil market, consumer demand for biofuels,
and government policies over time. It places a strong emphasis on the
behavior and decision-making of various economic agents among ten
geographic regions domestically. Although the BSM has historically been
used to develop insights into cellulosic ethanol industry growth and
market penetration, particularly with respect to policies and incentives
applicable to each supply-chain element (volumetric, capital, operating
subsidies; carbon caps/taxes; R\&D investment; loan guarantees; tax
credits), recent enhancements to the model allow it to treat the major
infrastructure-compatible fuels such as biomass-based gasoline, diesel,
and jet fuel. In general, scenario analysis based on the BSM shows that
the cellulosic ethanol industry tends not to rapidly thrive without
significant external actions in the early years of its evolution. An
initial focus for jumpstarting the industry typically has strongest
results in the BSM in areas where effects of intervention have been
identified to be multiplicative: due to industrial learning dynamics,
support for the construction of cellulosic ethanol conversion facilities
in the near future encourages the industry to flourish; in addition, the
alleviation of the bottlenecks of high-blend fuel distribution
infrastructure and high-blend fuel pump availability allows the
increased amount of ethanol produced to serve a viable market.

B. Bush, ``Biomass Scenario Model,'' presented at the 2009 Office of the
Biomass Program Analysis Platform Review, National Harbor, Maryland.
\textless{}http://www.obpreview2009.govtools.us/analysis/documents/FutureFuels1\_Bush.ppt\textgreater{}

B. Bush, ``Applications of the Biomass Scenario Model,'' presented at
the Workshop on Biofuels Projections in the AEO, Washington, D.C.
\textless{}http://www.eia.gov/biofuels/workshop/presentations/2013/pdf/presentation-14-032013.pdf\textgreater{}U.S.
policy targets 36 billion gallons per year of biofuels utilization by
2022, under the renewable fuels standard provisions of the Energy
Independence and Security Act of 2007. Achieving such large scale
biofuels adoption requires substantial development of new
infrastructure, markets, and related systems. The U.S. Department of
Energy is employing a system dynamics model, the Biomass Scenario Model
(BSM), to represent the primary system effects and dependencies in the
biomass-to-biofuels supply chain and to provide a framework for
developing scenarios and conducting biofuels policy analysis. This
approach is designed to help focus government action by determining
which supply chain changes would have the greatest potential to
accelerate the deployment of biofuels. Modeling the integration of all
aspects of the supply chain from growing the feedstock through harvest,
collection, transport, conversion, distribution of fuel and finally
consumption of the fuel in applicable vehicles (including the
availability of these vehicles) is critical to understanding where
government funds might be utilized most effectively. This presentation
provides an overview of the status of the BSM and a summary of recent
results from system analysis based on it. We find that policies which
are coordinated across the whole supply chain have significant impact in
fostering the growth of the biofuels industry.

S. Lee, R. George, and B. Bush, ``Estimating Solar PV Output Using
Modern Space/Time Geostatistics,'' presented at the 2009 Colorado
Renewable Energy Conference, Golden, Colorado.
\textless{}http://www.nrel.gov/docs/fy09osti/46208.pdf\textgreater{}This
presentation describes a project that uses mapping techniques to predict
solar output at subhourly resolution at any spatial point, develop a
methodology that is applicable to natural resources in general, and
demonstrate capability of geostatistical techniques to predict the
output of a potential solar plant.

\subsection{Patents}\label{patents}

Christopher L. Barrett, Richard J. Beckman, Keith A. Baggerly, Michael
D. McKay, Paul L. Speckman, Paula E. Stretz, Madhav V. Marathe, Stephen
G. Eubank, Brian W. Bush, James P. Smith, Katherine Campbell, Kathy P.
Berkbigler, Joerg Esser, Rudiger R. Jacob, Goran Konjevod, and Kai
Nagel, ``Urban Population Mobility Generation,''
\textless{}https://patentimages.storage.googleapis.com/pdfs/US20040088392.pdf\textgreater{}A
system and method provides a simulation of a complex network and
movement and interdependencies between entities in the network. The
system receives aggregated population data and a population synthesizer
generates disaggregated population data representative of two different
types of entities. The different entity types are then coupled to one
another to form interdependent relationships. An activity generator
generates typical activities for the entities. A route planner generates
travel plans, including departure times and travel modes, for each
entity to achieve daily activities. A micro-simulation module simulates
movement of the individual entities in compliance with their travel
plans. The system may include parallel processors to simulate thousands
of roadway and transit segments, intersection signals and signs,
transfer facilities between various transportation modes, traveler
origins and destinations, and entities and vehicles. The system includes
a framework and selector module that gathers the travel times from the
simulation and uses them to re-plan activities and trips and re-run the
simulation. The methods of the present invention produce appropriate
dynamic behavior of the transportation network as a whole.

\subsection{Magazine Articles}\label{magazine-articles}

T. W. Meyer, J. W. Davidson, I. G. Resnick, R. C. Gordon III, B. W.
Bush, C. Unal, G. L. Toole, L. J. Dowell, and S. S. Scott, ``The Los
Alamos Center for Homeland Security,'' Los Alamos Science, vol.~28,
pp.~192--197.

\subsection{Theses}\label{theses}

B. W. Bush, ``Track Reconstruction for Proton Decay,'' Senior Thesis,
California Institute of Technology, Pasadena, California. Events from
the 417 day IMB detector data sample are scanned visually to reconstruct
tracks present, and their possibility of coming from proton decay
assessed by considering their invariant mass and residual momentum.
Events from Monte Carlo simulations of neutrino background in the
detector are also analyzed similarly and compared to the IMB sample. No
significant signal above background has been found, so lifetime limits
for the four proton decay modes studied are presented.

B. W. Bush, ``Shape Fluctuations in Hot Rotating Nuclei,'' Doctor of
Philosophy, Yale University, New Haven, Connecticut.
\textless{}http://adsabs.harvard.edu/abs/1990PhDT\ldots{}\ldots{}.223B\textgreater{}We
present a unified theory of quadrupole shape fluctuations in highly
excited rotating nuclei using the framework of the Landau Theory of
shape transitions. The theory is applied to several experimental
observables. Our major application is the study of giant dipole
resonances (GDRs) built on hot rotating nuclei. With only two free
parameters, fixed by the ground state properties, the model reproduces
well experimental GDR cross-sections and angular correlations at any
temperature and spin in the 90 \textless{}= A \textless{}= 170 mass
range for both spherical and deformed nuclei. A systematic study of the
cross-section reveals that higher temperature cross-sections are
dominated by large fluctuations (triaxial in particular) and are less
sensitive to the equilibrium shape. To include non-adiabatic effects, we
generalize our theory to describe time-dependent shape fluctuations
using a stochastic approach based on the Langevin equation. This can
produce motional narrowing of the resonance. Comparisons with
experiments deviating from the adiabatic limit are used to determine the
damping of quadrupole motion at finite temperature. Another application
of the theory is in the study of E2 quasicontinuum spectra in warm
nuclei, where it predicts enhancement of the B(E2), in accord with the
experiment. Finally, we apply the fluctuation theory in improved
calculations of nuclear level densities as a function of energy and spin
using the static path approximation (SPA). Comparison with other
calculations and experiments are made.

\subsection{Reports}\label{reports}

F. J. Alexander, M. Anghel, K. Berkbigler, G. Booker, B. Bush, K. Davis,
A. Hoisie, N. Moss, S. Smith, T. P. Caudell, D. P. Holten, K. L.
Summers, and C. Zhou, ``Design, Implementation, and Validation of
Network and Workload Simulations for a 30-TeraOPS Computer System,'' Los
Alamos National Laboratory. The magnitude of the scientific computations
targeted by the US DOE ASCI project requires as-yet unavailable
computational power, and unprecedented bandwidth to enable remote,
realtime interaction with the compute servers. To facilitate these
computations ASCI plans to deploy massive computing platforms, possibly
consisting of tens of thousands of processors, capable of achieving
10-100 TeraOPS, with WAN connectivity from these to distant sites. For
various reasons the current approach to building a yet-larger
supercomputer--connecting commercially available SMPs with a
network--may be reaching practical limits. Better hardware design and
lower development costs require performance evaluation, analysis, and
modeling of parallel applications and architectures, and in particular
predictive capability. We outline an approach for simulating computing
architectures applicable to extreme-scale systems (thousands of
processors) and to advanced, novel architectural configurations, and
describe our progress in its realization. The simulation environment is
intended to allow (i) exploration of hardware/architecture design space;
(ii) exploration of algorithm/implementation space both at the
application level (e.g.~data distribution and communication) and the
system level (e.g.~scheduling, routing, and load balancing); (iii)
determining how application performance will scale with the number of
processors or other components; (iv) analysis of the tradeoffs between
performance and cost; and (v) testing and validating analytical models
of computation and communication. Our component-based design allows for
the seamless assembly of architectures from representations of workload,
processor, network interface, switches, etc. with disparate resolutions,
into an integrated simulation model. This accommodates different case
studies that may require different levels of fidelity in various parts
of a system. Our current implementation, includes low and
medium-fidelity models of the network and low-fidelity and direct
execution models of the workload. It supports studies of both simulation
performance and scaling, and the properties of the simulated system
themselves. Ongoing work allows more realistic simulation and dynamic
visualization of ASCI-like workloads on very large machines.

F. J. Alexander, K. Berkbigler, G. Booker, B. Bush, K. Davis, and A.
Hoisie, ``An Approach to Extreme-Scale Simulation of Novel
Architectures,'' Los Alamos National Laboratory, Report LA-UR-01-4087.
We outline an approach for simulating computing architectures applicable
to extreme-scale systems (thousands of processors) and to advanced,
novel architectural configurations. We believe that simulation is the
predictive tool of choice for evaluating the performance of such
systems. Our component-based design allows for the seamless assembly of
architectures from representations of workload, processor, network
interface, switches, etc., with disparate resolutions into an integrated
simulation model. This accommodates different case studies that may
require different levels of fidelity in various parts of a system. Our
initial prototype, comprising low-fidelity models of workload and
network, aims to model at least 4096 computational nodes in a fat-tree
network. It supports studies of simulation performance and scaling
rather than the properties of the simulated system themselves. Future
work will allow more realistic simulation and visualization of ASCI-like
workloads on very large machines.

F. J. Alexander, K. Berkbigler, G. Booker, B. Bush, K. Davis, A. Hoisie,
N. Moss, S. Smith, T. P. Caudell, D. P. Holten, K. L. Summers, and C.
Zhou, ``Design, Implementation, and Validation of Low- and
Medium-Fidelity Network Simulations of a 30-TeraOPS System,'' Los Alamos
National Laboratory, Report LA-UR-02-6573. The magnitude of the
scientific computations targeted by the US DOE ASCI project requires
as-yet unavailable computational power, and unprecedented bandwidth to
enable remote, realtime interaction with the compute servers. To
facilitate these computations ASCI plans to deploy massive computing
platforms, possibly consisting of tens of thousands of processors,
capable of achieving 10-100 TeraOPS, with WAN connectivity from these to
distant sites. For various reasons the current approach to building a
yet-larger supercomputer--connecting commercially available SMPs with a
network--may be reaching practical limits. Better hardware design and
lower development costs require performance evaluation, analysis, and
modeling of parallel applications and architectures, and in particular
predictive capability. We outline an approach for simulating computing
architectures applicable to extreme-scale systems (thousands of
processors) and to advanced, novel architectural configurations, and
describe our progress in its realization. The simulation environment is
intended to allow (i) exploration of hardware/architecture design space;
(ii) exploration of algorithm/implementation space both at the
application level (e.g.~data distribution and communication) and the
system level (e.g.~scheduling, routing, and load balancing); (iii)
determining how application performance will scale with the number of
processors or other components; (iv) analysis of the tradeoffs between
performance and cost; and (v) testing and validating analytical models
of computation and communication. Our component-based design allows for
the seamless assembly of architectures from representations of workload,
processor, network interface, switches, etc. with disparate resolutions,
into an integrated simulation model. This accommodates different case
studies that may require different levels of fidelity in various parts
of a system. Our current implementation, includes low and
medium-fidelity models of the network and low-fidelity and direct
execution models of the workload. It supports studies of both simulation
performance and scaling, and the properties of the simulated system
themselves. Ongoing work allows more realistic simulation and dynamic
visualization of ASCI-like workloads on very large machines.

F. J. Alexander, K. Berkbigler, G. Booker, B. Bush, K. Davis, A. Hoisie,
S. Smith, T. P. Caudell, D. P. Holten, K. L. Summers, and C. Zhou,
``Design and Implementation of Low- and Medium-Fidelity Network
Simulations of a 30-TeraOPS System,'' Los Alamos National Laboratory,
Report LA-UR-02-1930. The magnitude of the scientific computations
targeted by the ASCI project requires as-yet unavailable computational
power. To facilitate these computations ASCI plans to deploy massive
computing platforms, possibly consisting of tens of thousands of
processors, capable of achieving 10-100 TeraOPS. For various reasons the
current approach to building a yet-larger supercomputer--connecting
commercially available SMPs with a network--may be reaching practical
limits. The path to better hardware design and lower development costs
involves performance evaluation, analysis, and modeling of parallel
applications and architectures, and in particular predictive capability.
We outline an approach for simulating computing architectures applicable
to extreme-scale systems (thousands of processors) and to advanced,
novel architectural configurations. The proposed simulation environment
can be used for: (i) exploration of hardware/architecture design space;
(ii) exploration of algorithm/implementation space both at the
application level (e.g.~data distribution and communication) and the
system level (e.g.~scheduling, routing, and load balancing); (iii)
determining how application performance will scale with the number of
processors or other components; (iv) analysis of the tradeoffs between
performance and cost; and, (v) testing and validating analytical models
of computation and communication. Our component-based design allows for
the seamless assembly of architectures from representations of workload,
processor, network interface, switches, etc., with disparate
resolutions, into an integrated simulation model. This accommodates
different case studies that may require different levels of fidelity in
various parts of a system. Our initial implementation, comprising low-
and medium-fidelity models for the network and a low-fidelity model for
the workload, can simulate at least 4096 computational nodes in a
fat-tree network using Quadrics hardware. It supports studies of both
simulation performance and scaling, and the properties of the simulated
system themselves. Ongoing work allows more realistic simulation and
visualization of ASCI-like workloads on very large machines.

C. L. Barret, R. J. Beckman, K. P. Berkbigler, B. W. Bush, L. M. Moore,
and D. Visarraga, ``Actuated Signals in TRANSIMS,'' Los Alamos National
Laboratory, Report LA-UR-01-4609. This report outlines recent work
implementing and calibrating actuated traffic controls and vehicle
detectors in TRANSIMS. We have developed a generic control that provides
a flexible approach to representing such devices. Although not modeled
upon specific existing hardware or algorithms, our implementation
provides a responsive control over a wide variety of demand conditions.

C. L. Barrett, R. J. Beckman, K. P. Berkbigler, K. R. Bisset, B. W.
Bush, K. Campbell, S. Eubank, K. M. Henson, J. M. Hurford, D. A.
Kubicek, M. V. Marathe, J. Ramos, S. Ree, P. R. Romero, J. P. Smith, L.
L. Smith, P. L. Speckman, P. E. Stretz, G. L. Thayer, E. Van Eeckhout,
and M. D. Williams, ``TRANSIMS 2.0: Transportation Analysis Simulation
System: Volume 6 - Installation,'' Los Alamos National Laboratory,
Report LA-UR-00-1767.

C. L. Barrett, R. J. Beckman, K. P. Berkbigler, K. R. Bisset, B. W.
Bush, K. Campbell, S. Eubank, K. M. Henson, J. M. Hurford, D. A.
Kubicek, M. V. Marathe, J. Ramos, S. Ree, P. R. Romero, J. P. Smith, L.
L. Smith, P. L. Speckman, P. E. Stretz, G. L. Thayer, E. Van Eeckhout,
and M. D. Williams, ``TRANSIMS 2.0: Transportation Analysis Simulation
System: Volume 5 - Software Interface Functions and Data Structures,''
Los Alamos National Laboratory, Report LA-UR-00-1755.

C. L. Barrett, R. J. Beckman, K. P. Berkbigler, K. R. Bisset, B. W.
Bush, K. Campbell, S. Eubank, K. M. Henson, J. M. Hurford, D. A.
Kubicek, M. V. Marathe, J. Ramos, S. Ree, P. R. Romero, J. P. Smith, L.
L. Smith, P. L. Speckman, P. E. Stretz, G. L. Thayer, E. Van Eeckhout,
and M. D. Williams, ``TRANSIMS 2.0: Transportation Analysis Simulation
System: Volume 4 - Calibrations, Scenarios, and Tutorials,'' Los Alamos
National Laboratory, Report LA-UR-00-1766.

C. L. Barrett, R. J. Beckman, K. P. Berkbigler, K. R. Bisset, B. W.
Bush, K. Campbell, S. Eubank, K. M. Henson, J. M. Hurford, D. A.
Kubicek, M. V. Marathe, J. Ramos, S. Ree, P. R. Romero, J. P. Smith, L.
L. Smith, P. L. Speckman, P. E. Stretz, G. L. Thayer, E. Van Eeckhout,
and M. D. Williams, ``TRANSIMS 2.0: Transportation Analysis Simulation
System: Volume 3 - Modules,'' Los Alamos National Laboratory, Report
LA-UR-00-1725.

C. L. Barrett, R. J. Beckman, K. P. Berkbigler, K. R. Bisset, B. W.
Bush, K. Campbell, S. Eubank, K. M. Henson, J. M. Hurford, D. A.
Kubicek, M. V. Marathe, J. Ramos, S. Ree, P. R. Romero, J. P. Smith, L.
L. Smith, P. L. Speckman, P. E. Stretz, G. L. Thayer, E. Van Eeckhout,
and M. D. Williams, ``TRANSIMS 2.0: Transportation Analysis Simulation
System: Volume 2 - Networks and Vehicles,'' Los Alamos National
Laboratory, Report LA-UR-00-1724.

C. L. Barrett, R. J. Beckman, K. P. Berkbigler, K. R. Bisset, B. W.
Bush, K. Campbell, S. Eubank, K. M. Henson, J. M. Hurford, D. A.
Kubicek, M. V. Marathe, J. Ramos, S. Ree, P. R. Romero, J. P. Smith, L.
L. Smith, P. L. Speckman, P. E. Stretz, G. L. Thayer, E. Van Eeckhout,
and M. D. Williams, ``TRANSIMS 2.0: Transportation Analysis Simulation
System: Volume 1 - Technical Overview,'' Los Alamos National Laboratory,
Report LA-UR-00-1723.

C. L. Barrett, R. J. Beckman, K. P. Berkbigler, K. R. Bisset, B. W.
Bush, K. Campbell, S. Eubank, K. M. Henson, J. M. Hurford, D. A.
Kubicek, M. V. Marathe, J. Ramos, S. Ree, P. R. Romero, J. P. Smith, L.
L. Smith, P. L. Speckman, P. E. Stretz, G. L. Thayer, E. Van Eeckhout,
and M. D. Williams, ``TRANSIMS 2.0: Transportation Analysis Simulation
System: Volume 7 - Methods in TRANSIMS,'' Los Alamos National
Laboratory, Report LA-UR-02-4217.

C. L. Barrett, R. J. Beckman, K. P. Berkbigler, K. R. Bisset, B. W.
Bush, K. Campbell, S. Eubank, K. M. Henson, J. M. Hurford, D. A.
Kubicek, M. V. Marathe, P. R. Romero, J. P. Smith, L. L. Smith, P. E.
Stretz, G. L. Thayer, E. Van Eeckhout, and M. D. Williams, ``TRANSIMS
Portland Study Reports: 8. Appendix: Scripts, Configuration Files,
Special Travel Time Function,'' Los Alamos National Laboratory, Report
LA-UR-01-5716.

C. L. Barrett, R. J. Beckman, K. P. Berkbigler, K. R. Bisset, B. W.
Bush, K. Campbell, S. Eubank, K. M. Henson, J. M. Hurford, D. A.
Kubicek, M. V. Marathe, P. R. Romero, J. P. Smith, L. L. Smith, P. E.
Stretz, G. L. Thayer, E. Van Eeckhout, and M. D. Williams, ``TRANSIMS
Portland Study Reports: 5. Postprocessing for Environmental Analysis,''
Los Alamos National Laboratory, Report LA-UR-01-5715.

C. L. Barrett, R. J. Beckman, K. P. Berkbigler, K. R. Bisset, B. W.
Bush, K. Campbell, S. Eubank, K. M. Henson, J. M. Hurford, D. A.
Kubicek, M. V. Marathe, P. R. Romero, J. P. Smith, L. L. Smith, P. E.
Stretz, G. L. Thayer, E. Van Eeckhout, and M. D. Williams, ``TRANSIMS
Portland Study Reports: 4. General Results,'' Los Alamos National
Laboratory, Report LA-UR-01-5714.

C. L. Barrett, R. J. Beckman, K. P. Berkbigler, K. R. Bisset, B. W.
Bush, K. Campbell, S. Eubank, K. M. Henson, J. M. Hurford, D. A.
Kubicek, M. V. Marathe, P. R. Romero, J. P. Smith, L. L. Smith, P. E.
Stretz, G. L. Thayer, E. Van Eeckhout, and M. D. Williams, ``TRANSIMS
Portland Study Reports: 3. Feedback Loops,'' Los Alamos National
Laboratory, Report LA-UR-01-5713.

C. L. Barrett, R. J. Beckman, K. P. Berkbigler, K. R. Bisset, B. W.
Bush, K. Campbell, S. Eubank, K. M. Henson, J. M. Hurford, D. A.
Kubicek, M. V. Marathe, P. R. Romero, J. P. Smith, L. L. Smith, P. E.
Stretz, G. L. Thayer, E. Van Eeckhout, and M. D. Williams, ``TRANSIMS
Portland Study Reports: 2. Study Setup: Parameters and Input Data,'' Los
Alamos National Laboratory, Report LA-UR-01-5712.

C. L. Barrett, R. J. Beckman, K. P. Berkbigler, K. R. Bisset, B. W.
Bush, K. Campbell, S. Eubank, K. M. Henson, J. M. Hurford, D. A.
Kubicek, M. V. Marathe, P. R. Romero, J. P. Smith, L. L. Smith, P. E.
Stretz, G. L. Thayer, E. Van Eeckhout, and M. D. Williams, ``TRANSIMS
Portland Study Reports: 1. Introduction/Overiew,'' Los Alamos National
Laboratory, Report LA-UR-01-5711.

C. L. Barrett, R. J. Beckman, K. P. Berkbigler, K. R. Bisset, B. W.
Bush, S. Eubank, J. M. Hurford, G. Konjevod, D. A. Kubicek, M. V.
Marathe, J. D. Morgeson, M. Rickert, P. R. Romero, L. L. Smith, M. P.
Speckman, P. L. Speckman, P. E. Stretz, G. L. Thayer, and M. D.
Williams, ``TRANSIMS (TRansportation ANalysis SIMulation System) 1.0:
Volume 6 - Installation,'' Los Alamos National Laboratory, Report
LA-UR-99-2580.

C. L. Barrett, R. J. Beckman, K. P. Berkbigler, K. R. Bisset, B. W.
Bush, S. Eubank, J. M. Hurford, G. Konjevod, D. A. Kubicek, M. V.
Marathe, J. D. Morgeson, M. Rickert, P. R. Romero, L. L. Smith, M. P.
Speckman, P. L. Speckman, P. E. Stretz, G. L. Thayer, and M. D.
Williams, ``TRANSIMS (TRansportation ANalysis SIMulation System) 1.0:
Volume 3 - Files,'' Los Alamos National Laboratory, Report
LA-UR-99-2579.

C. L. Barrett, R. J. Beckman, K. P. Berkbigler, K. R. Bisset, B. W.
Bush, S. Eubank, J. M. Hurford, G. Konjevod, D. A. Kubicek, M. V.
Marathe, J. D. Morgeson, M. Rickert, P. R. Romero, L. L. Smith, M. P.
Speckman, P. L. Speckman, P. E. Stretz, G. L. Thayer, and M. D.
Williams, ``TRANSIMS (TRansportation ANalysis SIMulation System) 1.0:
Volume 2 - Software, Part 5 - Libraries,'' Los Alamos National
Laboratory, Report LA-UR-99-2578.

C. L. Barrett, R. J. Beckman, K. P. Berkbigler, K. R. Bisset, B. W.
Bush, S. Eubank, J. M. Hurford, G. Konjevod, D. A. Kubicek, M. V.
Marathe, J. D. Morgeson, M. Rickert, P. R. Romero, L. L. Smith, M. P.
Speckman, P. L. Speckman, P. E. Stretz, G. L. Thayer, and M. D.
Williams, ``TRANSIMS (TRansportation ANalysis SIMulation System) 1.0:
Volume 2 - Software, Part 4 - Tools,'' Los Alamos National Laboratory,
Report LA-UR-99-2577.

C. L. Barrett, R. J. Beckman, K. P. Berkbigler, K. R. Bisset, B. W.
Bush, S. Eubank, J. M. Hurford, G. Konjevod, D. A. Kubicek, M. V.
Marathe, J. D. Morgeson, M. Rickert, P. R. Romero, L. L. Smith, M. P.
Speckman, P. L. Speckman, P. E. Stretz, G. L. Thayer, and M. D.
Williams, ``TRANSIMS (TRansportation ANalysis SIMulation System) 1.0:
Volume 2 - Software, Part 3 - Test Networks,'' Los Alamos National
Laboratory, Report LA-UR-99-2576.

C. L. Barrett, R. J. Beckman, K. P. Berkbigler, K. R. Bisset, B. W.
Bush, S. Eubank, J. M. Hurford, G. Konjevod, D. A. Kubicek, M. V.
Marathe, J. D. Morgeson, M. Rickert, P. R. Romero, L. L. Smith, M. P.
Speckman, P. L. Speckman, P. E. Stretz, G. L. Thayer, and M. D.
Williams, ``TRANSIMS (TRansportation ANalysis SIMulation System) 1.0:
Volume 2 - Software, Part 2 - Selectors,'' Los Alamos National
Laboratory, Report LA-UR-99-2575.

C. L. Barrett, R. J. Beckman, K. P. Berkbigler, K. R. Bisset, B. W.
Bush, S. Eubank, J. M. Hurford, G. Konjevod, D. A. Kubicek, M. V.
Marathe, J. D. Morgeson, M. Rickert, P. R. Romero, L. L. Smith, M. P.
Speckman, P. L. Speckman, P. E. Stretz, G. L. Thayer, and M. D.
Williams, ``TRANSIMS (TRansportation ANalysis SIMulation System) 1.0:
Volume 2 - Software, Part 1 - Modules,'' Los Alamos National Laboratory,
Report LA-UR-99-2574.

C. L. Barrett, R. J. Beckman, K. P. Berkbigler, K. R. Bisset, B. W.
Bush, S. Eubank, J. M. Hurford, G. Konjevod, D. A. Kubicek, M. V.
Marathe, J. D. Morgeson, M. Rickert, P. R. Romero, L. L. Smith, M. P.
Speckman, P. L. Speckman, P. E. Stretz, G. L. Thayer, and M. D.
Williams, ``TRANSIMS (TRansportation ANalysis SIMulation System) 1.0:
Volume 0 - Overview,'' Los Alamos National Laboratory, Report
LA-UR-99-1658.

C. L. Barrett, K. B. Berkbigler, K. R. Burris, B. W. Bush, S. D. Hull,
J. M. Hurford, P. Medvick, D. A. Kubicek, M. Marathe, J. D. Morgeson, K.
Nagel, D. J. Roberts, L. L. Smith, M. J. Stein, P. E. Stretz, S. J.
Sydoriak, K. Cervenka, M. Morris, and R. Donnelly, ``The Dallas-Ft.
Worth Case Study,'' Los Alamos National Laboratory, Report
LA-UR-97-4502.

C. Barrett, R. Beckman, K. Berkbigler, K. Burris, B. Bush, R. Donnelly,
S. Hull, J. Hurford, D. Kubicek, P. Medvick, K. Nagel, D. Roberts, L.
Smith, M. Stein, P. Stretz, and S. Sydoriak, ``Transportation Analysis
Simulation System (TRANSIMS) Version 1.0 User Notebook,'' Los Alamos
National Laboratory, Report LA-UR-98-848.

N. Becker and B. Bush, ``Metropolitan CIP/DSS Key Resources Sector
Model,'' Los Alamos National Laboratory, LA-UR-04-5314.

N. Becker, L. Bettencourt, W. Buehring, W. Beyeler, T. Brown, B. Bush,
S. Conrad, L. Dauelsberg, S. DeLand, C. Joslyn, P. Kaplan, R. LeClaire,
V. Loose, M. North, D. Powell, S. Rasmussen, K. Saeger, M. Samsa, D.
Thompson, R. Whitfield, and M. Witkowski, ``Guidelines for Determining
CIP/DSS Infrastructure Model Depth and Breadth Requirements,'' Los
Alamos National Laboratory, Report LA-UR-04-5324.

N. Becker, W. Buehring, W. Beyeler, T. Brown, B. Bush, S. Conrad, L.
Dauelsberg, S. DeLand, M. Ebinger, S. Folga, G. Hirsch, P. Kaplan, J.
Kavicky, V. Koritarov, R. LeClaire, Z. Li, V. Loose, M. McLamore, D.
Newsom, E. Portante, D. Powell, S. Rasmussen, K. Saeger, D. Sallach, M.
Samsa, S. Shamsuddin, J. St.~Aubin, D. Thompson, R. Whitfield, and M.
Witkowski, ``CIP/DSS Gap Analysis,'' Los Alamos National Laboratory,
Report LA-UR-04-5325.

R. J. Beckman, K. P. Berkbigler, B. W. Bush, and P. Stretz, ``TRANSIMS:
Portland study calibration of river crossing screen lines,'' Los Alamos
National Laboratory, Report LA-UR-01-1921.

R. J. Beckman, B. W. Bush, K. M. Henson, and P. E. Stretz, ``Portland
Study Synthetic Population,'' Los Alamos National Laboratory, Report
LA-UR-01-4610. TRANSIMS (Transportation Analysis and Simulation System)
is an integrated system of travel forecasting models designed to give
transportation planners accurate, complete information on traffic
impacts, congestion, and pollution. The Population Synthesizer Module
constructs a regional population imitation with demographics closely
matching the real population. Households are distributed spatially to
approximate regional population distribution. The synthetic population's
demographics form basis for individual and household activities
requiring travel and their household locations determine some of the
travel origins and destinations. This report outlines how we have
constructed the synthetic population for our Portland, Oregon, case
study. It also briefly summarizes the characteristics of the data, and
how we verified that the data was correctly generated.

R. Bent, B. Bhaduri, D. Billingsley, A. Boissonnade, J. Bossert, R.
Bowne, M. Brown, A. Burris, B. Bush, J. Coen, C. Davis, J. Doyle, R.
Erickson, M. Ewers, S. Fernandez, P. Fitzpatrick, J. Florez, A. Ganguly,
G. Geernaert, E. Gilleland, R. Gislason, F. Griffith, R. Haut, K.
Henson, G. Holland, M. Kramer, R. LeClaire, R. Linn, R. Lopez, A. Lynch,
L. Margolin, J. Maslanik, D. O'Brien, D. Parsons, D. Pasqualini, P.
Patelli, W. Priedhorsky, E. Regnier, T. Ringler, J. Rush, P. Sheng, S.
Swerdlin, E. Van Eeckhout, R. Wagoner, S. Walden, T. Warner, J. Wegiel,
P. Welsh, L. Wilder, B. Wolshon, and Y. Zhang, ``Recommendations for
Research on Extreme Weather Impacts on Infrastructure,'' Workshop on
Weather Extremes Impacts on Infrastructure.

R. Bent, B. Bhaduri, D. Billingsley, A. Boissonnade, J. Bossert, R.
Bowne, M. Brown, A. Burris, B. Bush, J. Coen, C. Davis, J. Doyle, R.
Erickson, M. Ewers, S. Fernandez, P. Fitzpatrick, J. Florez, A. Ganguly,
G. Geernaert, E. Gilleland, R. Gislason, F. Griffith, R. Haut, K.
Henson, G. Holland, M. Kramer, R. LeClaire, R. Linn, R. Lopez, A. Lynch,
L. Margolin, J. Maslanik, D. O'Brien, D. Parsons, D. Pasqualini, P.
Patelli, W. Priedhorsky, E. Regnier, T. Ringler, J. Rush, P. Sheng, S.
Swerdlin, E. Van Eeckhout, R. Wagoner, S. Walden, T. Warner, J. Wegiel,
P. Welsh, L. Wilder, B. Wolshon, and Y. Zhang, ``Précis of
Recommendations for Research on Extreme Weather Impacts on
Infrastructure,'' Workshop on Weather Extremes Impacts on
Infrastructure.

K. P. Berkbigler and B. W. Bush, ``TRANSIMS Network Subsystem for
IOC-1,'' Los Alamos National Laboratory, Report LA-UR-97-1580. The
TRANSIMS network representation provides access to detailed information
about streets, intersections, and signals in a road network. It forms a
layer separating the other subsystems from the actual network data
tables so that the other subsystems do not need to access the data
tables directly or deal with the format and organization of the tables.
This subsystem allows the user to construct multiple subnetworks from
the network database tables. It includes road network objects such as
nodes (intersections), links (road/street segments), lanes, and traffic
controls (signs and signals).

K. P. Berkbigler and B. W. Bush, ``TRANSIMS simulation output subsystem
for IOC-1,'' Los Alamos National Laboratory, Los Alamos, New Mexico,
Technical Report LA-UR--97-1226.
\textless{}http://www.osti.gov/energycitations/servlets/purl/501501-8MefLN/webviewable/\textgreater{}The
output subsystem collects data from a running microsimulation, stores
the data for future use, and manages the subsequent retrieval of the
data. It forms a layer separating the other subsystems from the actual
data files so that the other subsystems do not need to access the data
files at the physical level or deal with the physical location and
organization of the files. This subsystem also allows the user to
specify what data is collected and retrieved, and to filter it by space
and time. The collection occurs in a distributed manner such that the
subsystem`s impact on the microsimulation performance is minimized; the
retrieval provides a unified view of the distributed data.

K. P. Berkbigler, B. W. Bush, and J. F. Davis, ``TRANSIMS Software
Architecture for IOC-1,'' Los Alamos National Laboratory, Report
LA-UR-97-1242. This document describes the TRansportation ANalysis
SIMulation System (TRANSIMS) software architecture and high-level design
for the first Interim Operational Capability (IOC-1). Our primary goal
in establishing the TRANSIMS software architecture is to lay down a
framework for IOC-1. We want to make sure that the various components of
TRANSIMS are effectively integrated, both for IOC-1 and beyond, so that
TRANSIMS remains flexible, expandable, portable, and maintainable
throughout its lifetime. In addition to outlining the high-level design
of the TRANSIMS software, we also set forth the software development
environment and software engineering practices used for TRANSIMS.

K. Berkbigler, B. Bush, and K. Davis, ``À la carte,'' Los Alamos
National Laboratory, Report LA-UR-01-5735/LALP-01-243.

K. Berkbigler, B. Bush, K. Davis, A. Hoisie, S. Smith, C. Zhou, K.
Summers, and T. Caudell, ``Graph Visualization for the Analysis of the
Structure and Dynamics of Extreme-Scale Supercomputers,'' Los Alamos
National Laboratory, Report LA-UR-02-1929. We are exploring the
development and application of information visualization techniques for
the analysis of new extreme-scale supercomputer architectures. Modern
super-computers typically comprise very large clusters of commodity SMPs
interconnected by possibly dense and often nonstandard networks. The
scale, complexity, and inherent nonlocality of the structure and
dynamics of this hardware, and the systems and applications distributed
over it, challenge traditional analysis methods. As part of the a la
carte team at Los Alamos National Laboratory, who are simulating these
advanced architectures, we are exploring advanced visualization
techniques and creating tools to provide intuitive exploration,
discovery, and analysis of these simulations. This work complements
existing and emerging algorithmic analysis tools. Here we gives
background on the problem domain, a description of a prototypical
computer architecture of interest (on the order of 10,000 processors
connected by a quaternary fat-tree network), and presentations of
several visualizations of the simulation data that make clear the flow
of data in the interconnection network.

A. Berscheid and B. Bush, ``Critical Infrastructure Protection Decision
Support System Metropolitan Models: PHASE IV Validation Report,'' Los
Alamos National Laboratory, LA-UR-05-1599.

A. Berscheid and B. Bush, ``Critical Infrastructure Protection Decision
Support System Models and Simulations: Bioterrorism,'' Los Alamos
National Laboratory.

A. Berscheid, W. Beyeler, B. Bush, L. Dauelsberg, S. DeLand, A. Fishman,
A. Ivey, M. Jusko, L. Moore, D. Powell, L. Olson, R. Richardson, J.
St.~Aubin, M. Samsa, and D. Thompson, ``CIP/DSS Phase IV Architecture
and Analysis Process Status Report,'' Los Alamos National Laboratory,
LA-UR-06-0538.

A. Berscheid, B. Bush, L. Dauelsberg, A. Ivey, D. Thompson, L. Moore, D.
Powell, W. Beyeler, S. DeLand, A. Fishman, M. Jusko, J. St.~Aubin, L.
Olson, and M. Samsa, ``CIP/DSS Architecture and Analysis Process Status
Report,'' Los Alamos National Laboratory, LA-UR-05-5846.

M. P. Blue and B. W. Bush, ``Set Entropy of Block Configurations That
Appear in the TRANSIMS Simulation,'' Los Alamos National Laboratory,
Report LA-UR-01-4277.

M. P. Blue and B. W. Bush, ``Critical Energy Infrastructure Contingency
Screening Heuristics Status Report,'' Los Alamos National Laboratory,
Report LA-UR-04-7649.

M. Blue, B. Bush, and J. Puckett, ``Applications of Fuzzy Logic to Graph
Theory,'' Los Alamos National Laboratory, Report LA-UR-96-4792. Graph
theory has numerous applications to problems in systems analysis,
operations research, transportation, and economics. In many cases,
however, some aspects of the graph-theoretic problem are uncertain. In
these cases, it can be useful to deal with this uncertainty using the
methods of fuzzy logic. This paper discusses the taxonomy of fuzzy
graphs, formulates some standard graph-theoretic problems (shortest
paths, maximum flow, minimum cut, and articulation points) in terms of
fuzzy graphs, and provides algorithmic solutions to these problems, with
examples.

T. Brown, B. Bush, S. DeLand, and D. Powell, ``CIP/DSS Model Development
Plan,'' Los Alamos National Laboratory, LA-UR-04-6147.

B. Bush, ``Energy Infrastructure Modeling at Los Alamos National
Laboratory,'' Los Alamos National Laboratory, Report
LALP-03-027/LA-UR-03-0658.

B. Bush, ``Metropolitan CIP/DSS Energy Sector Model,'' Los Alamos
National Laboratory, Report LA-UR-04-5322.

B. Bush, ``Critical Infrastructure Analysis for Extreme Weather
Events,'' Los Alamos National Laboratory.

B. W. Bush, ``A Tool for Drawing Undirected Graphs,'' Los Alamos
National Laboratory, Report LA-UR-96-2166. The problem of laying out, or
drawing, a graph arises in a wide variety of contexts. Automatically
drawing computer network (e.g., LAN or WAN) configurations,
object-oriented class diagrams, or database entity-relationship diagrams
are examples of graph drawing. Estimating the layout of street networks
containing some intersections with unknown locations is an example of
the problem of drawing a graph. This paper discusses an algorithm for
drawing general undirected graphs that relies on constructing a
dynamical system analogous to the graph and evolving the state of the
system to an equilibrium configuration. This configuration provides an
aesthetically pleasing layout of the graph. We present an implementation
of the algorithm as a C++ class. We also demonstrate the use of the
class in a command-line executable program compilable on a variety of
computer platforms as well as in an interactive 32-bit Windows program
that animates the layout process.

B. W. Bush, ``TRANSIMS Database Subsystem for IOC-1,'' Los Alamos
National Laboratory, Report LA-UR-97-987. The TRANSIMS database
subsystem provides low-level services for accessing and modifying
TRANSIMS data. It forms a layer separating the other subsystems from the
actual data files so that the other subsystems do not need to access the
data files at the physical level or deal with the physical location and
organization of the files. This subsystem also organizes the data and
supports a variety of metadata. It uses a relational model for the
storage of data.

B. W. Bush, ``TRANSIMS Input Editor System for IOC-1,'' Los Alamos
National Laboratory, Report LA-UR-97-1642. The TRANSIMS input editor
provides a means for managing the TRANSIMS database, editing road
network data, and setting up scenarios for simulation via its graphical
user interface (GUI). It separates the user from the lower-level layers
of TRANSIMS software involved with data management. It has functions for
manipulating data in the TRANSIMS database; for creating, importing,
altering, validating, and viewing road network data; and for setting up
simulation output tables. The input editor is integrated into the
ArcView geographic information system (GIS) and the Oracle relational
database. One can also customize or extend the input editor using the
Avenue programming language.

B. W. Bush, ``NISAC Interdependent Energy Infrastructure Simulation
System,'' Los Alamos National Laboratory, LA-UR-04-7700.

B. W. Bush, ``CIP/DSS `Conductor' Tutorial,'' Los Alamos National
Laboratory, LA-UR-06-3459.

B. W. Bush, ``Conductor Tutorial,'' Los Alamos National Laboratory,
LA-UR-06-3459.

B. W. Bush, ``Extreme Weather Coupled to Infrastructure Damage,'' Los
Alamos National Laboratory.

B. W. Bush, ``Simulating Crisis Behavior,'' Los Alamos National
Laboratory.

B. W. Bush, ``Global Social-Technical Systems Simulation: Addressing the
Methodological Challenge,'' Los Alamos National Laboratory.

B. W. Bush, ``TRANSIMS and the hierarchical data format,'' Los Alamos
National Laboratory, Los Alamos, New Mexico, Technical Report
LA-UR--97-2240.
\textless{}http://www.osti.gov/energycitations/servlets/purl/516007-rkPCoV/webviewable/\textgreater{}The
Hierarchical Data Format (HDF) is a general-purposed scientific data
format developed at the National Center for Supercomputing Applications.
It supports metadata, compression, and a variety of data structures
(multidimensional arrays, raster images, tables). FORTRAN 77 and ANSI C
programming interfaces are available for it and a wide variety of
visualization tools read HDF files. The author discusses the features of
this file format and its possible uses in TRANSIMS.

B. W. Bush, ``Notes on object-orientation,'' Los Alamos National
Laboratory, Los Alamos, New Mexico, Technical Report LA-UR--96-3020.
\textless{}http://www.osti.gov/energycitations/servlets/purl/369658-L8v8FU/webviewable/\textgreater{}This
report discusses the uses of programming by object-orientation. Included
in this report are the following: overview of concepts; software
development; user interfaces; and databases.

B. W. Bush and J. R. Nix, ``New Approach to the Interaction of Cosmic
Rays with Nuclei in Spacecraft Shielding and the Human Body,'' Los
Alamos National Laboratory, Report LA-12452-MS. The interaction of
high-energy cosmic rays with nuclei in spacecraft shielding and the
human body is important for manned interplanetary missions and is not
well understood either experimentally or theoretically. We present a new
theoretical approach to this problem based on classical hadrodynamics
for extended nucleons, which treats nucleons of finite size interacting
with massive meson fields. This theory represents the classical analogue
of the quantum hadrodynamics of Serot and Walecka without the
assumptions of the mean-field approximation and point nucleons. It
provides a natural covariant microscopic approach to collisions between
cosmic rays and nuclei that automatically includes space-time
non-locality and retardation, nonequilibrium phenomena, interactions
among all nucleons, and particle production. Unlike previous models,
this approach is manifestly Lorentz covariant and satisfies a priori the
basic conditions that are present when cosmic rays collide with nuclei,
namely an interaction time that is extremely short and a nucleon
mean-free path, force range, and internucleon separation that are all
comparable in size. We review the history of classical meson-field
theory and derive the classical relativistic equations of motion for
nucleons of finite size interacting with massive scalar and vector meson
fields.

B. W. Bush and R. D. Small, ``Smoke Produced by Nonurban Target-Area
Fires Following a Nuclear Exchange,'' Pacific-Sierra Research
Corporation, Report 1515. The amount of smoke that may be produced by
wildland or rural fires as a consequence of a large-scale nuclear
exchange is estimated. The calculation is based on a compilation of
rural military facilities, identified from a wide variety of
unclassified sources, together with data on their geographic positions,
surrounding vegetation (fuel), and weather conditions. The ignition area
(corrected for fuel moisture) and the amount of fire spread are used to
calculate the smoke production. The results show a substantially lower
estimated smoke production (from wildland fires) than in earlier nuclear
winter studies. The amount varies seasonally and at its peak is less by
an order of magnitude that the estimated threshold level necessary for a
major attenuation of solar radiation.

B. W. Bush and R. D. Small, ``Nuclear Winter Source-Term Studies: A
Preliminary Analysis of Soviet Urban Areas,'' Pacific-Sierra Research
Corporation, Report 1628.

B. W. Bush and R. D. Small, ``Nuclear Winter Source-Term Studies: The
Classification of U. S. Cities,'' Pacific-Sierra Research Corporation,
Report 1628. A theory for classifying U.S. cities according to their
burnable densities is developed. Urban land use, which is closely
related to combustible loadings, is shown to be a classification
correlate superior to the conventional measures of city rank such as
population, urban area, or population density. Six classes of cities are
defined. The basic division is regional and the classification is shown
to account for the demographic and economic characteristics that
distinguish U.S. urban areas. Estimates of smoke production based on
analysis of sample cities from each group would systematically account
for differences in urban geographies.

B. W. Bush and R. D. Small, ``Nuclear Winter Source-Term Studies:
Ignition of Silo-Field Vegetation by Nuclear Weapons,'' Pacific-Sierra
Research Corporation, Report 1628. Smoke produced by the ignition and
burning of live vegetation by nuclear explosions has been suggested as a
major contributor to a possible nuclear winter. This report considers
the mechanics of live vegetation ignition by a finite-radius nuclear
fireball. For specified plant properties, the amount of fireball
radiation absorbed by a plant community is calculated as a function of
depth into the stand and range from the fireball. The spectral regions
of plant energy absorption and the overlap with the emitted fireball
thermal spectra are discussed. A simple model for the plant response to
the imposed thermal load is developed. First, the temperature is raised;
the change depends on the plant structure, moisture content, and plant
canopy. Subsequent energy deposition desiccates the plant and finally
raises its temperature to the threshold ignition limit. Results show the
development of a variable depth ignition zone. Close to the fireball,
ignition of the entire plant occurs. At greater distances (several
fireball radii) portions of the plant are only partially desiccated, and
sustained burning is less probable. Far from the burst, the top of the
stand is weakly heated, and only a small transient temperature change
results. An estimate of the smoke produced by an exchange involving the
U.S. missile fields shows that the burning of live vegetation only
slightly increases the total nonurban smoke production.

B. W. Bush and R. D. Small, ``Smoke produced by nonurban target-area
fires following a nuclear exchange. Technical report, 5 June 1984-5
January 1985,'' Pacific-Sierra Research Corporation, Los Angeles,
California, Technical Report PSR--1515.
\textless{}http://www.osti.gov/energycitations/product.biblio.jsp?osti\_id=5054780\textgreater{}The
amount of smoke that may be produced by wildland or rural fires as a
consequence of a large-scale nuclear exchange is estimated. The
calculation is based on a compilation of rural military facilities,
identified from a wide variety of unclassified sources, together with
data on their geographic positions, surrounding vegetation (fuel), and
weather conditions. The ignition area (corrected for fuel moisture) and
the amount of fire spread are used to calculate the smoke production.
The results show a substantially lower estimated smoke production (from
wildland fires) than in earlier nuclear-winter studies. The amount
varies seasonally and at its peak is less by an order of magnitude that
the estimated threshold level necessary for a major attenuation of solar
radiation.

B. W. Bush, K. P. Berkbigler, and L. L. Smith, ``TRANSIMS Data
Preparation Guide,'' Los Alamos National Laboratory, Report
LA-UR-98-1411.

B. W. Bush, M. A. Dore, G. H. Anno, and R. D. Small, ``Nuclear Winter
Source-Term Studies: Smoke Produced by a Nuclear Attack on the United
States,'' Pacific-Sierra Research Corporation, Report 1628.

B. W. Bush, C. R. Files, and D. R. Thompson, ``Empirical
Characterization of Infrastructure Networks,'' Los Alamos National
Laboratory, Report LA-UR-01-5784. Critical infrastructure protection is
a recognized problem of national importance. Infrastructure networks
such as electric power, natural gas, communications, and transportation
systems have an inherent graph-theoretic structure. Quantitatively
characterizing the essential properties of infrastructure networks for
various domains lays a valuable foundation for studying the universal
features (especially criticality, robustness, etc.) and specific
characteristics of such networks. We construct an extensive reference
data set of infrastructure network graphs: 44 graphs of 13 types with
nearly one million vertices and over one million edges. After
regularizing these graphs, we compute more than fifty metrics related to
connectivity, distance scale, cyclicity, cliquishness, and redundancy.
We contrast these metrics for different types of infrastructures, study
their interrelationship, and use them to cluster and classify systems.
We consider both intact networks and networks that have been degraded by
the removal of some vertices or edges either at random or
systematically--this provides insight as to the robustness of the
network if it were subject to a natural disaster or an attack.

B. W. Bush, L. M. Ransohoff, and R. D. Small, ``Target Area Studies:
Smoke Produced by a Nuclear Attack on the Soviet Union,'' Pacific-Sierra
Research Corporation, Report 1842.

B. Bush, G. Anno, R. McCoy, R. Gaj, and R. Small, ``Nuclear Winter
Source-Term Studies: Fuel Loads in U. S. Cities,'' Pacific-Sierra
Research Corporation, Report 1628.

B. Bush, S. DeLand, and M. Samsa, ``Critical Infrastructure Protection
Decision Support System (CIP/DSS) Project Overview,'' Los Alamos
National Laboratory, Report LA-UR-04-5319.

B. Bush, P. Giguere, J. Holland, S. Linger, A. McCown, M. Salazar, C.
Unal, D. Visarraga, K. Werley, R. Fisher, S. Folga, M. Jusko, J.
Kavicky, M. McLamore, E. Portante, and S. Shamsuddin, ``Interdependent
Energy Infrastructure Simulation System (IEISS) Software Manual, Version
1.0,'' Los Alamos National Laboratory, Report LA-UR-03-1317.

B. Bush, P. Giguere, J. Holland, S. Linger, A. McCown, M. Salazar, C.
Unal, D. Visarraga, K. Werley, R. Fisher, S. Folga, M. Jusko, J.
Kavicky, M. McLamore, E. Portante, and S. Shamsuddin, ``Interdependent
Energy Infrastructure Simulation System (IEISS) Technical Reference
Manual, Version 1.0,'' Los Alamos National Laboratory, Report
LA-UR-03-1318.

B. Bush, P. Giguere, J. Holland, S. Linger, A. McCown, M. Salazar, C.
Unal, D. Visarraga, K. Werley, R. Fisher, S. Folga, M. Jusko, J.
Kavicky, M. McLamore, E. Portante, and S. Shamsuddin, ``Interdependent
Energy Infrastructure Simulation System (IEISS) User Manual, Version
1.0,'' Los Alamos National Laboratory, Report LA-UR-03-1319.

B. Bush, T. Jenkin, D. Lipowicz, D. Arent, and R. Cooke, ``Variance
Analysis of Wind and Natural Gas Generation under Different Market
Structures: Some Observations,'' National Renewable Energy Laboratory,
Golden, Colorado, Research Report TP-6A20-52790.
\textless{}http://www.nrel.gov/docs/fy12osti/52790.pdf\textgreater{}Does
large scale penetration of renewable generation such as wind and solar
power pose economic and operational burdens on the electricity system? A
number of studies have pointed to the potential benefits of renewable
generation as a hedge against the volatility and potential escalation of
fossil fuel prices. Research also suggests that the lack of correlation
of renewable energy costs with fossil fuel prices means that adding
large amounts of wind or solar generation may also reduce the volatility
of system-wide electricity costs. Such variance reduction of system
costs may be of significant value to consumers due to risk aversion. The
analysis in this report recognizes that the potential value of risk
mitigation associated with wind generation and natural gas generation
may depend on whether one considers the consumer's perspective or the
investor's perspective and whether the market is regulated or
deregulated. We analyze the risk and return trade-offs for wind and
natural gas generation for deregulated markets based on hourly prices
and load over a 10-year period using historical data in the PJM
Interconnection (PJM) from 1999 to 2008. Similar analysis is then
simulated and evaluated for regulated markets under certain assumptions.

B. Bush, M. Melaina, M. Penev, and W. Daniel, ``SERA Scenarios of Early
Market Fuel Cell Electric Vehicle Introductions: Modeling Framework,
Regional Markets, and Station Clustering,'' National Renewable Energy
Laboratory, Golden, Colorado, Technical Report NREL/TP-5400-56588.
\textless{}http://www.nrel.gov/docs/fy13osti/56588.pdf\textgreater{}The
availability of fueling infrastructure has become a major barrier to the
early market success of hydrogen fuel cell electric vehicles (FCEVs).
Various models have addressed infrastructure development during the
early transition phase, but few long-term models have captured
development dynamics in a manner that is consistent with real-world
planning activities. This report describes the development and analysis
of detailed temporal and spatial scenarios for early market
infrastructure clustering and vehicle rollout using the Scenario
Evaluation, Regionalization and Analysis (SERA) model. The scenarios
reconcile nationwide scenario dynamics from a National Academy of
Sciences study (NAS 2008) with observations and lessons learned from
California's early market strategy and planning activities (CaFCP 2012).
The report provides an overview of the SERA scenario development
framework and discusses the approach used to develop the nationwide
scenario. The capability to focus on detailed infrastructure rollout
dynamics within particular regions and states is then discussed with
reference to Northeast Corridor states. The report also provides a
description of the enhanced station placement algorithms developed to
simulate both urban area network coverage and station clustering in
neighborhoods with high densities of early adopters. Results from the
national scenario analysis suggest that long-term levelized delivered
costs for hydrogen tend toward \$6.00/kg nationally, and zero cumulative
cash flow is achieved in about 2018 or 2025 if hydrogen is priced at
\$11.00/kg and \$6.75/kg, respectively. The capability to focus on
dynamics within particular regions and to articulate detailed station
placement strategies within urban areas adds realism and a planning
perspective to these national scenario results.

B. Bush, L. Ransohoff, R. McCoy, and R. Small, ``Target Area Studies:
Nuclear Winter Source Terms for Soviet Laydowns,'' Pacific-Sierra
Research Corporation, Report 1842.

R. H. Byrne, B. W. Bush, and T. Jenkin, ``Long-term Modelling of Natural
Gas Prices,'' Sandia National Laboratories, SAND2013-2898. There are
many factors that influence the price of natural gas. These include
weather forecasts, economic activity, storage inventory, market
expectations, and in the longer term supply and demand fundamentals.
These factors can also influence the price volatility on a variety of
timescales. While accurately predicting natural gas prices over long
periods is probably futile, there are several reasons for modeling
future long-term prices. First, business decisions for long-term
investments, e.g.~whether or not to invest in a power plant that burns
natural gas, require estimates of future prices over a multi-decade time
horizon. Second, price paths, probability density functions, and
volatility estimates are necessary to price different types of
derivative products. A third example, which was the motivation for this
effort, is that estimating the future uncertainty of electricity prices
over a multi-decade horizon under different generation mix scenarios
requires some sort of estimate of input prices for natural gas and other
fossil fuels. There are several options for modeling long-term price
movements of natural gas. One is to develop a multi-factor model,
develop longterm estimates of the factors, and then use these to
construct the expected price path. Another option is to fit a
mean-reverting stochastic model to historical data. Both approaches have
pitfalls. Developing accurate long-term estimates of factors that
contribute to natural gas prices is virtually impossible because of the
inability to predict unforseen events. By fitting a stochastic model to
historical data, one is assuming that the distribution of prices in the
future will match the past. This is often a false assumption.
Distributions (and correlations) of price dynamics often change over
time, and the past is not necessarily a good predictor of the future.
The approach taken for this effort was a stochastic ⬚fit to historical
data. To incorporate some uncertainty into the price paths, the model
accurately replicates historical distributions about 58\% of the time.
If there is a strong belief that the future distributions will match
historical distributions, an acceptance-testing method is outlined for
generating price paths that perfectly match the distribution of
historical data.

CIP/DSS Team, ``Critical Infrastructure Decision Support System
(CIP/DSS) Biological Capability Case Study Summary Report,'' Los Alamos
National Laboratory.

J. Darby, B. Bush, S. Eisenhawer, and T. Bott, ``Methodology for
Optimizing Allocation of Resources to Protect Infrastructure against
Acts of Terrorism,'' Los Alamos National Laboratory, LA-UR 04-0590.

M. A. Dore, B. W. Bush, G. H. Anno, and R. D. Small, ``Nuclear Winter
Source-Term Studies: Urban Area Analysis,'' Pacific-Sierra Research
Corporation, Report 1628.

J. Hacker, J. Boehnert, and B. Bush, ``Using Ensemble NWP Output for
Infrastructure Impact Forecasts: First Steps,'' National Center for
Atmospheric Research.

D. Inman, L. Vimmerstedt, E. Newes, B. Bush, and S. Peterson, ``Biomass
scenario model scenario library: definitions, construction, and
description,'' National Renewable Energy Laboratory, Golden, Colorado,
Technical Report.
\textless{}http://dx.doi.org/10.2172/1129277\textgreater{}Understanding
the development of the biofuels industry in the United States is
important to policymakers and industry. The Biomass Scenario Model (BSM)
is a system dynamics model of the biomass-to-biofuels system that can be
used to explore many aspects of the industry. Because of the complexity
of the model, as well as the wide range of possible future conditions
that affect biofuels industry development, we have not developed a
single reference case but instead have designed a set of six
incentive-focused scenarios. The purpose of this report is to describe
the scenarios that comprise the BSM scenario library. At present, we
have the following six incentive-focused scenarios in our library:
minimal incentives scenario; ethanol-focused incentives scenario; equal
access to incentives scenario; output-focused incentives scenario;
pathway-diversity-focused incentives scenario; and the
point-of-production-focused incentives scenario. This report describes
the model settings and rationale for each scenario.

A. Ivey, L. Toole, B. Bush, and S. Fernandez, ``Summary of Critical
Infrastructure Hurricane-Impact Estimation Methodology,'' Los Alamos
National Laboratory, LA-UR-06-1042.

A. Ivey, L. Toole, B. Bush, and S. Fernandez, ``Summary of Critical
Infrastructure Hurricane-Impact Estimation Methodology,'' Los Alamos
National Laboratory, LA-UR-06-1042.

T. Jenkin, V. Diakov, E. Drury, B. Bush, P. Denholm, J. Milford, D.
Arent, R. Margolis, and R. Byrne, ``The Use of Solar and Wind as a
Physical Hedge against Price Variability within a Generation
Portfolio,'' National Renewable Energy Laboratory, Golden, Colorado,
Technical Report NREL/TP-6A20-59065.
\textless{}http://www.nrel.gov/docs/fy13osti/59065.pdf\textgreater{}This
study provides a framework to explore the potential use and incremental
value of small- to large-scale penetration of solar and wind
technologies as a physical hedge against the risk and uncertainty of
electricity cost on multi-year to multi-decade timescales. Earlier
studies characterizing the impacts of adding renewable energy (RE) to
portfolios of electricity generators often used a levelized cost of
energy or simplified net cash flow approach. In this study, we expand on
previous work by demonstrating the use of an 8760 hourly production cost
model (PLEXOS) to analyze the incremental impact of solar and wind
penetration under a wide range of penetration scenarios for a region in
the Western U.S. We do not attempt to ``optimize'' the portfolio in any
of these cases. Rather we consider different RE penetration scenarios,
that might for example result from the implementation of a Renewable
Portfolio Standard (RPS) to explore the dynamics, risk mitigation
characteristics and incremental value that RE might add to the system.
We also compare the use of RE to alternative mechanisms, such as the use
of financial or physical supply contracts to mitigate risk and
uncertainty, including consideration of their effectiveness and
availability over a variety of timeframes.

LANL CIP/DSS Team, ``CIP/DSS Metropolitan Sector Diagrams,'' Los Alamos
National Laboratory, LA-UR-05-3364.

Y. Lin, E. Newes, B. Bush, S. Peterson, and D. Stright, ``Biomass
Scenario Model Documentation: Data and References,'' National Renewable
Energy Laboratory, Golden, Colorado, Technical Report
NREL/TP-6A20-57831.
\textless{}http://www.osti.gov/bridge/servlets/purl/1082565/\textgreater{}The
Biomass Scenario Model (BSM) is a system dynamics model that represents
the entire biomass-to-biofuels supply chain, from feedstock to fuel use.
The BSM is a complex model that has been used for extensive analyses;
the model and its results can be better understood if input data used
for initialization and calibration are well-characterized. It has been
carefully validated and calibrated against the available data, with data
gaps filled in using expert opinion and internally consistent assumed
values. Most of the main data sources that feed into the model are
recognized as baseline values by the industry. This report documents
data sources and references in Version 2 of the BSM (BSM2), which only
contains the ethanol pathway, although subsequent versions of the BSM
contain multiple conversion pathways. The BSM2 contains over 12,000
total input values, with 506 distinct variables. Many of the variables
are opportunities for the user to define scenarios, while others are
simply used to initialize a stock, such as the initial number of
biorefineries. However, around 35\% of the distinct variables are
defined by external sources, such as models or reports. The focus of
this report is to provide insight into which sources are most
influential in each area of the supply chain. We find that data based on
POLYSYS datasets and U.S. Department of Agriculture baseline projections
are the most utilized sources in the feedstock sector, whereas the
conversion module relies heavily on data found in National Renewable
Energy Laboratory technical reports dealing with the techno-economic
characteristics of different technologies. The distribution, dispensing,
and fuel use modules utilize data on gasoline stations from the National
Association of Convenience Stores.

L. M. Ransohoff, G. H. Anno, B. W. Bush, and R. D. Small, ``Topics in
Nuclear Winter Source-Term Research: Composition of Residential
Structures in the United States,'' Pacific-Sierra Research Corporation,
Report 1761.

L. Ransohoff, K. Knudsen, B. Bush, and R. Small, ``Target Area Studies:
Material Inventory and Smoke Properties,'' Pacific-Sierra Research
Corporation, Report 1842.

M. Samsa, R. Raynor, S. M. DeLand, H.-S. J. Min, D. R. Powell, W. E.
Beyeler, G. Hirsch, R. Whitfield, J. Fair, L. Dauelsberg, B. W. Bush,
and R. J. LeClaire, ``Critical Infrastructure Protection Decision
Support System Evaluation of a Biological Scenario.'' Sandia National
Laboratories, SAND2005-2399C.
\textless{}http://www.osti.gov/scitech/biblio/966930\textgreater{}

R. D. Small, B. W. Bush, and M. A. Dore, ``Topics in Nuclear Winter
Source-Term Research: Initial Smoke Distribution for Nuclear Winter
Calculations,'' Pacific-Sierra Research Corporation, Report 1761.

D. Thompson, D. Powell, and B. Bush, ``CIP/DSS Metropolitan Team
Standardization Guide,'' Los Alamos National Laboratory, LA-UR-04-6148.

C. Unal, K. Werley, P. Giguere, B. Bush, and R. Gordon, ``Energy
Interdependence Simulation,'' Los Alamos National Laboratory, Report
LA-UR-01-5297.

C. Unal, K. Werley, P. Giguere, B. Bush, A. Bersheid, R. Gordon, and F.
Roach, ``The SOFIA Project for Interdependent Infrastructure Modeling,
Simulation, and Analysis: Modeling Approach, Program Description, and
Performance,'' Los Alamos National Laboratory, Report LA-UR-01-1658.

L. J. Vimmerstedt, B. W. Bush, and S. Peterson, ``Effects of Deployment
Investment on the Growth of the Biofuels Industry,'' National Renewable
Energy Laboratory, Golden, Colorado, Technical Report.
\textless{}http://dx.doi.org/10.2172/1118095\textgreater{}In support of
the national goals for biofuel use in the United States, numerous
technologies have been developed that convert biomass to biofuels. Some
of these biomass-to-biofuel conversion technology pathways are fully
commercial, while others are in earlier stages of development. The
advancement of a new pathway towards commercialization involves various
types of improvements, including yield improvements through chemical and
biochemical refinements, process engineering, and financial performance.
Actions of private investors and public programs can accelerate the
demonstration and deployment of new conversion technology pathways.
These investors (both private and public) will pursue a range of pilot-,
demonstration-, and pioneer-commercial-scale biorefinery investments,
because the most cost-effective set of investments for advancing the
maturity of the pathway is unknown. In some cases whether or not the
pathway itself will ultimately be technically and financially successful
is unknown. This report presents results from the Biomass Scenario
Model---a system dynamics model of the biomass-to-biofuels system---that
estimate effects of investment in one particular demonstration and
deployment plan. This plan is a multi-stage combination of pilot,
demonstration, and pioneer-commercial-scale biorefineries. The report
discusses challenges in estimating effects of such investments. The
report concludes that investment in demonstration and deployment appears
to have a substantive positive effect on the development of the biofuels
industry, and that other conditions, such as supportive policies, are
likely to have major impacts on the effectiveness of such investments.

A. A. Zagonel, B. W. Bush, S. H. Conrad, S. M. DeLand, I. J.
Martinez-Moyano, and D. Thompson, ``Emergent Methodological
Contributions from Modeling for Critical Infrastructure Protection,''
Sandia National Laboratory.

\subsection{Software}\label{software}

K. P. Berkbigler, B. W. Bush, K. Campbell, S. Eubank, D. A. Kubicek, D.
J. Roberts, P. R. Romero, J. P. Smith, P. Stretz, and M. D. Williams,
TRANSIMS. Los Alamos National Laboratory.

G. B. Booker, B. W. Bush, P. T. Giguere, J. V. Holland, S. P. Linger, M.
L. Salazar, C. Unal, and K. A. Werley, Interdependent Energy
Infrastructure Simulation System. Los Alamos National Laboratory. The
IEISS software models energy transmission network systems (such as
electric power systems and natural gas pipelines) and simulates their
physical behavior, including the interdependencies between systems (such
as when the energy supplied by one system is used to operate components
of another system). Each physical, logical, or functional entity in the
model has a variety of attributes and behaviors that mimic its
real-world counterpart. The software supports the analysis of the
complex, non-linear, and emergent interactions between energy
infrastructures at the state, regional, or national scale. (Databases
are not supplied with the software, however.) Specifically, the
simulation can be used to visualize the interconnectivity between
different energy systems, predict the outcome of incidents affecting the
networks, measure the economic effects of disruptions in service, assess
system robustness under varied future plans and forecasts, and identify
components critical for the operation of the systems.

B. W. Bush, A Tool for Drawing Undirected Graphs.

B. W. Bush, T. Cleland, L. Lauer, and D. R. Thompson, CIP/DSS
``Conductor'' Tool. Los Alamos National Laboratory. The ``Conductor'' is
a Java-based tool to merge source code from Vensim (a system dynamics
application code available from Ventana Systems, http://www.vensim.com)
into a single file. Its main use is to merge separate modules of a
larger system into a single version which can be executed by Vensim. It
also has functionality to perform simple checks for software coding
conventions, and model browsing. The software also contains extensive
support for database connectivity.

B. W. Bush, L. R. Dauelsberg, M. H. Ebinger, R. J. LeClaire, D. R.
Powell, S. Rasmussen, D. R. Thompson, C. J. Wilson, M. S. Witkowski, A.
Ford, and D. Newsom, Metropolitan Critical Infrastructure Model. Los
Alamos National Laboratory. The Metropolitan Critical Infrastructure
Model simulates the dynamics of fourteen critical infrastructures
(agriculture, banking/finance, chemical industry, defense industrial
base, emergency services, food, government,
information/telecommunications, key resources, postal, public health
transportation, water, energy) in urban areas at a highly aggregate
level (i.e., total capacities/capabilities are represented instead of
individual facilities). The purpose of the models is to simulate
disruption scenarios, evaluate the consequences of such disruptions, and
estimate the effectiveness of mitigation actions. The models include
high-level infrastructure interdependencies, damage and recovery
simulations, potential lost productivity and recovery cost, and
aggregate market models. Dynamic processes like these are represented in
the CIP/DSS infrastructure sector simulations by differential equations,
discrete events, and codified rules of operation. The consequences are
computed in terms of human health and safety, economic, public
confidence, national security, and environmental impacts. Realistic
databases are supplied separately from the software. The system is
designed to help answer the following questions: (i) What are the
consequences of attacks on infrastructure in terms of national security,
economic impact, public health, and conduct of government---including
the consequences that propagate to other infrastructures? (ii) Are there
choke points in our Nation's infrastructures (i.e., areas where one or
two attacks could have the largest impact)? What and where are the choke
points? (iii) Incorporating consequence, vulnerability, and threat
information into an overall risk assessment, what are the highest risk
areas? (iv) What investment strategies can the U.S. make that will have
the most impact in reducing overall risk?

\subsection{Videos}\label{videos}

G. B. Booker, B. W. Bush, R. C. Gordon, J. F. Roach, C. Unal, M. A.
Wolinsky, W. Buehring, R. Fisher, S. Folga, and J. Peerenboom,
``Infrastructure Interdependence Simulation Prototype Video,'' Los
Alamos National Laboratory, Video LA-UR-00-2245.

\end{document}